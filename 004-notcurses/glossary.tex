\newglossaryentry{mojibake}
{
  name={文字化け (mojibake [{\fontspec{DejaVu Serif}mod͡ʑibake}])},
  description={The garbled result of mismatched
    character encoding and decoding. Japanese, from 文字 (moji (character))
    化ける (bakeru (take a different form)). Russians know it as \textrussian{кракозя́бры}
    (krakozyabry [{\fontspec{DejaVu Serif}krɐkɐˈzʲæbrɪ̈}]) or sometimes
    \textrussian{бНОПНЯ} (bnopnjá {\fontspec{DejaVu Serif}bnɐpˈnʲa}), while Bulgarians speak of \textbulgarian{маймуница}
    (majmunica (monkey alphabet)). Serbs cut to a characteristically
    Balkan chase with \textrussian{ђубре} (đubre (trash)). I {\fontspec{DejaVu Serif}�} Unicode}
}

\newglossaryentry{smcup}
{
  name={alternate screen},
  description={A capability of some terminals (described by the \texttt{smcup}
    termcap capability and \texttt{enter\_ca\_mode} terminfo capability). 
    Entering the alternate screen sees the screen cleared, and output will not
    be added to the scrollback buffer. Output to the alternate screen is no
    longer visible upon leaving it. Whether or not the original screen contents
    are restored is terminal dependent; if the \texttt{non\_rev\_rmcup}
    terminfo capability is defined, the original screen contents will \textit{not}
    be restored.}
}

\newglossaryentry{ansix364}
{
  name={ANSI X3.64-1979},
  description={``Control Sequences for Video Terminals and Peripherals''.
    Ratified \Gls{ecma48}}
}

\newglossaryentry{ascii}
{
  name={ANSI X3.4-1968},
  description={``Coded character set---7-bit American National Standard Code
    for Information Interchange''. A 128-element codeset and a 7-bit character
    encoding perhaps better known as ASCII\cite{ascii}. Its code set is a
    proper subset of \Gls{ucs}, and its encoding is a proper subset of
    \gls{utf8} (but \textit{not} \gls{utf16}, nor \gls{utf32}). In terms of
    \gls{standardsets}, ASCII is the union of \Gls{isoiec646}'s
    dotriaconta-character \Gls{C0} control set (ISO registration 001) and
    a 94-character graphics set (ISO registration 006)}
}

\newglossaryentry{ASCII}
{
  name={ASCII},
  description={See \Gls{ascii}}
}

\newglossaryentry{canonical mode}
{
  name={canonical mode},
  description={See \gls{cooked mode}}
}

\newglossaryentry{cooked mode}
{
  name={cooked mode},
description={The default terminal mode under SUSv4 (see \gls{raw mode}).
  The terminal driver buffers input until a newline is entered, while echoing
  it to the screen. Ctrl+C is mapped to \texttt{SIGINT}, Ctrl+\textbackslash\ is
  mapped to \texttt{SIGQUIT}, and Ctrl+Z is mapped to \texttt{SIGTSTP}.
  Buffered input is flushed when these signals are sent. Also known as
  \gls{canonical mode}}
}

\newglossaryentry{cdk}
{
  name={CDK},
description={The Curses Development Kit, a widget library developed against the
  Curses specification. It is not distributed as part of NCURSES, though
  both are maintained by Thomas E.\ Dickey}
}

\newglossaryentry{cell}
{
  name={cell},
description={The rectilinear area corresponding to a given text-mode coordinate.
  The analogue of pixels in graphics mode, a cell typically contains more
  pixels along its height than its width. 12 lines of 6 pixels each is not
  unreasonable. A cell can usually represent a grapheme cluster, a foreground
  color, a background color, and some number of attributes, independently
  from surrounding cells}
}

\longnewglossaryentry{character}
{
  name={character},
description={One of the most imprecise terms in computer science. The C and \CC
  languages have a data type \texttt{char}, which usually cannot hold the
  world's characters. Whether it is signed or unsigned is implementation-defined.
  Should I speak of them, I'll write \texttt{char}. A character is an evanescence;
  it is easier to say what it is \text{not}:
  \begin{denseitemize}
  \item{A character is not necessarily unique within a character set (see diacritic vs.\ precomposed forms).}
  \item{Different characters needn't be distinct glyphs\cite{nothinggoesaway}.}
  \item{A character needn't limit itself to a single column, even when using a fixed-width font.}
  \item{A character does not necessarily have a visual representation.}
  \item{A character does not necessarily have a single meaning among different languages.}
  \item{In a given encoding, all characters are not necessarily the same size.}
  \item{A character cannot necessarily capture the state of the keyboard at some time.}
  \item{A character cannot necessarily describe a cell of a display at some time.}
  \end{denseitemize}

  In the absence of a coded character set or character encoding for context, a
  character is really nothing more than an identifier, e.g. \texttt{LATIN
    CAPITAL LETTER I}. Isn't that a recursive definition? You betcha.

  When I use the word ``character'', I mean either an element of some code set, or
  such an element as encoded by some character encoding. The context ought make
  obvious which meaning is intended}
}

\newglossaryentry{characterencoding}
{
  name={character encoding},
  description={A mapping from one or more \glspl{codeset} to a set of octet
  sequences. Also known as a \textbf{character encoding scheme}\cite{rfc2978}}
}

\newglossaryentry{codepoint}
{
  name={code point},
description={A numerical value within a code space. In and of itself,
  it has neither an associated sequence of bits nor any particular glyph.
  Indeed, a code point might correspond to some non-printable control}
}

\newglossaryentry{codeset}
{
  name={code set},
description={A set of abstract characters, each having a name and a code
  point. \Gls{isoiec10646} specifies almost 150,000 abstract characters as the
  Universal Coded Character Set (UCS). This is smaller than its corresponding
  code space by about two orders of magnitude. Also known as a
  \textbf{coded character set}\cite{rfc2978}}
}

\newglossaryentry{codespace}
{
  name={code space},
description={A numerical range spanning all the code points of a code set.
  The Unicode code space (as of Unicode 13) is made up of 17 contiguous
  planes of 65,536 contiguous code points each, for a size of $17*2^{16}$}
}

\newglossaryentry{console}
{
  name={console},
description={Though sometimes used synonymously with ``terminal'', console almost
  always refers to text-based environments on personal computers. On Linux
  x86 machines, this could mean a serial terminal, an AT keyboard and a VGA
  in 80x25 text mode, a USB keyboard with a UEFI framebuffer, or any number
  of other things. When I use ``console'', I mean terminals outside of a GUI
  context}
}

\newglossaryentry{curses}
{
  name={Curses},
description={The X/OSI specification of an API for optimized cursor routines. It
  was most recently ratified as part of the January 2018 Single UNIX
  Specification version 4 administrative rollup, aka ``susv4-2018''. NCURSES is
  an implementation of Curses (plus extensions). Notcurses is not}
}

\newglossaryentry{cursor}
{
  name={cursor},
description={Both the location where output will next be written, and possibly
  a visual indication of that location}
}

\newglossaryentry{dialog}
{
  name={dialog},
description={An NCURSES-based application for TUI modal dialogs}
}

\newglossaryentry{directcolor}
{
  name={DirectColor},
  description={FIXME},
}

\newglossaryentry{ebcdic}
{
  name={EBCDIC},
  description={IBM's Extended Binary Coded Decimal Interchange Codes, announced
    alongside the legendary System/360, extended the BCDICs into a collection
    of 8-bit codes. Not a single code matches its ASCII equivalent, in part
    because EBCDIC has properties including a discontiguous Latin alphabet.
    EBCDIC is thankfully rare in modern practice, living on primarily as
    a boogieman with which the owners of standard library factories frighten
    students learning C. Exhaustive information regarding EBCDIC can be found
    in \cite{codedcharactersets}}
}

\newglossaryentry{egc}
{
  name={extended grapheme cluster},
description={FIXME }
}

\newglossaryentry{ecma6}
{
  name={ECMA-6:1991},
  description={``7-bit Coded Character Set'', 6th edition. Ratified
    \Gls{isoiec646}}
}

\newglossaryentry{ecma35}
{
  name={ECMA-35:1994},
  description={``Character Code Structure and Extension Techniques'', 6th
    edition. In the words of Mike Frysinger, it ``covers some fundamentals
    related to character sets, the notions of \Gls{C0}, C1, G0 (but not their
    contents), and how escape sequences work (but not what they mean)\cite{hterm}.''
    Ratified \Gls{isoiec2022}}
}

\newglossaryentry{ecma43}
{
  name={ECMA-43:1991},
    description={``8-bit Coded Character Set Structure and Rules'', 3rd edition. Extends
      \Gls{ecma6}, conforms to \Gls{ecma35}. Also published as \Gls{isoiec4873}}
}

\newglossaryentry{ecma48}
{
  name={ECMA-48:1991},
  description={``Control Functions for Coded Character Sets'', 5th edition.
    Control functions and their coded representations for use with ECMA-35-compliant
    7-bit and 8-bit codes.
   Ratified as \Gls{isoiec6429}}
}

\newglossaryentry{ecma94}
{
  name={ECMA-94:1986},
  description={``8-bit Single Byte Coded Graphic Character Sets---Latin
    Alphabets \textnumero\ 1 to \textnumero\ 4.'', 2nd edition. Ratified
    \Gls{isoiec8859}-1, -2, -3, and -4}
}

\newglossaryentry{gamma correction}
{
  name={gamma correction},
  description={Compression and decompression based around the nonlinear nature
   of human color perception. The eye has more sensitivity to differences between
   darker shades than lighter ones.}
}

\newglossaryentry{hsv}
{
  name={HSV},
  description={The Hue, Saturation, and Value color model. A transformation of \Gls{rgb}}
}

\newglossaryentry{isoiec646}
{
  name={ISO/IEC 646:1991},
  description={``Information technology---ISO 7-bit coded character set for
    information interchange''. A seven-bit encoding developed in conjunction
    with ASCII. Of its 128 codes, 116 are \texttt{invariant characters}, and
    the remaining 12 are defined by national ISO 646 variants. All variants
    have 95 graphic characters (space and 94 printables), and 33 control
    characters (these definitions give rise to ANSI C's \texttt{ctype}
    definitions). The only full 128-character encoding defined in the document
    is the International Reference Version; as of the most recent (1991)
    edition, the IRV is explicitly identical to ASCII. ISO/IEC 646:1991 is
    twinned with \Gls{itut50}, and ratified as \Gls{ecma6}}
}

\newglossaryentry{isoiec2022}
{
  name={ISO/IEC 2022:1994},
  description={``Information technology---Character code structure and extension techniques''.
   An absolute mess of a standard. FIXME
   Ratified as \Gls{ecma35}}
}

\newglossaryentry{isoiec6429}
{
  name={ISO/IEC 6429:1992},
  description={``Information technology---Control functions for coded character sets''.
   Ratified \Gls{ecma48}}
}

\newglossaryentry{isoiec8613}
{
  name={ISO/IEC 8613-6:1994},
  description={``Information technology---Open Document Architecture (ODA) and Interchange Format: Character content architectures''.
    Defines many graphic renditions in terms of \Gls{sgr}. Referenced by \Gls{ecma48} in terms
    of future standardization (for \Gls{sgr} attributes 38 and 48, sometimes
    mistakenly called ``ANSI color''). Twinned to \Gls{itut416}}
}

\newglossaryentry{isoiec8859}
{
  name={ISO/IEC 8859:1998},
  description={``Information technology---8-bit single-byte coded graphic character sets''.
    A set of 15 standards, emerging from 1987 to 2001, of which ISO 8859-1:1987 (Part 1: Latin alphabet \textnumero\ 1) is
    probably the best-known. 8859 covers only graphic characters, not control
    characters. In 1992, ISO-8859-1 (note the extra hyphen compared to ISO 8859-1; egads),
    was registered, mating ISO 8859-1 with \Gls{isoiec646}'s \Gls{C0} and
    \Gls{isoiec6429}'s C1 for a complete 256-code encoding. This result is a
    proper subset of UTF-8, and a proper superset of ASCII. The first four were
    ratified as \Gls{ecma94}}
}

\newglossaryentry{isoiec4873}
{
  name={ISO/IEC 4873:1991},
  description={``Information technology---ISO 8-bit code for information interchange---Structure and rules for implementation''.
    Extends \Gls{isoiec646}. Defines three ``levels'' of 8-bit encodings, which
    you can read about if you want to (I don't recommend it). Level 1 is
    realized by \Gls{isoiec8859}. Levels 2 and 3 are realized by \Gls{isoiec10367}.
    Also published as \Gls{ecma43}}
}

\newglossaryentry{isoiec10367}
{
  name={ISO/IEC 10367:1991},
  description={``Information technology---Standardized coded graphic character sets for use in 8-bit codes''.
    Realizes Layers 2 and 3 of the \Gls{isoiec4873} specification by defining
    G0, G1, G2, and G3 sets}
}

\newglossaryentry{isoiec2375}
{
  name={ISO/IEC 2375:2003},
  description={``Information technology---Procedure for registration of escape sequences and coded character sets''.
   Policies governing registration for \Gls{isoiec646} and \Gls{isoiec2022}}
}

\newglossaryentry{standardsets}
{
  name={standard character sets},
  description={Any character set registered in the \textit{ISO Register of Coded
               Characters to be Used With Escape Sequences} (ISO-IR) via the
             provisions of \Gls{isoiec2375}. ISO standard character sets are
             either ``control sets'' or ``graphic sets'' per \Gls{isoiec2022}}
}

\newglossaryentry{C0}
{
  name={C0},
  description={The first 32 values of the 7-bit \Gls{isoiec646}, including
    \texttt{NUL}, \texttt{BEL}, \texttt{BS}, \texttt{HT}, \texttt{LF},
    \texttt{VT}, \texttt{FF}, and \texttt{CR}. ISO registration number 001}
}

\newglossaryentry{iso88591}
{
  name={ISO-8859-1},
  description={IANA preferred name for ``Latin alphabet~\textnumero\ 1'' from ISO/IEC 8859-1,
    with \Gls{C0} and C1 from \Gls{isoiec6429}. Quoth Frank da Cruz:
    ``Similarly, ISO 8859-1 Latin Alphabet 1, which most of us think of as a
    coherent 8-bit character set is, in truth, composed of the \Gls{isoiec646}
    control set (registration 001), the ISO 646 USA graphics set (006), the
    characters Space and Delete, a second 32-character control set (normally,
    but not necessarily, \Gls{isoiec6429}, registration 077), and a 96-character set
    known as ``The Right-hand Part of Latin Alphabet 1'' (registration 100). Each of
    these pieces except Space and Delete has its own unique registration number
    and designating escape sequence. There is no single, unique identifier for
    the 8-bit Latin-1 character set in its entirety\cite{interchange}.''
  }
}

\newglossaryentry{isoiec10646}
{
  name={ISO/IEC 10646:2017},
  description={``Information technology---Universal Coded Character Set (UCS)''.
    FIXME
    }
}

\newglossaryentry{itut416}
{
  name={ITU T.416},
  description={Twinned to \Gls{isoiec8613}}
}

\newglossaryentry{itut50}
{
  name={ITU T.50},
  description={The International Reference Alphabet (IRA), formerly known as
   International Alphabet \textnumero\ 5. Its most recent (1992)
   edition is twinned to \Gls{isoiec646}:1991, and thus (despite is name)
   leaves 12 positions open for national variants}
}

\newglossaryentry{keyboard}
{
  name={keyboard},
  description={An input device primarily used to select from a set of
  characters. These characters might or might not correspond to elements in
  the active encoding. Unlike pointers, keyboards do not indicate position}
}

\newglossaryentry{ncurses}
{
  name={NCURSES},
description={An implementation of X/OSI Curses (plus extensions) maintained by
  Thomas E.\ Dickey under the auspices of the Free Software Foundation.
  Installed on just about every currently-running UNIX system, it is a
  venerable, proven library, and the foundation of a great many applications}
}

\newglossaryentry{newt}
{
  name={newt},
description={A minimal TUI library written atop slang, providing the foundation
  for \texttt{whiptail}. It is not a Curses implementation}
}

\newglossaryentry{noncanonical}
{
  name={non-canonical mode},
description={See \gls{raw mode}}
}

\newglossaryentry{notcurses}
{
  name={Notcurses},
description={The baddest character graphics/TUI library in town}
}

\newglossaryentry{pointer}
{
  name={pointer},
description={An input device primarily used to indicate position. It
  corresponds with one cell at any given time, which might or might not be
  independent of the cursor. There can be more than one pointer. Like a
  \gls{keyboard}, pointers typically have one or more distinct buttons---though
  usually far fewer than a keyboard}
}

\newglossaryentry{pgroup}
{
  name={process group},
description={A set of processes sharing the same PGID. When the terminal
  generates a signal for a process (e.g. \texttt{SIGINT} in response to Ctrl+C), it
  sends the signal to all processes in that process group. Of the process
  groups making up a session, only one is the foreground process group.
  Attempts to perform terminal I/O by session processes outside the foreground
  process group will result in \texttt{SIGTTIN} or \texttt{SIGTTOU}}
}

\longnewglossaryentry{pseudoterminal}
{
  name={pseudoterminal},
description={``A pair of virtual character devices that provide a
  bidirectional communication channel. One end of the channel is called the
  master; the other end is called the slave. The slave end of the
  pseudoterminal provides an interface that behaves exactly like a classical
  terminal. A process that expects to be connected to a terminal, can open
  the slave end of a pseudoterminal and then be driven by a program that has
  opened the master end. Anything that is written on the master end is
  provided to the process on the slave end as though it was input typed on a
  terminal.'' (\texttt{pty(7)}, \textit{Linux Programmer's Manual}\cite{pty7})

 Pseudoterminals are used by terminal emulators and interactive SSH
  connections. They are essentially buffers with some basic transformation
  capabilities. Ctrl-C is turned into \texttt{SIGINT} by the pseudoterminal device,
  and it handles relevant \texttt{ioctl()}s}
}

\newglossaryentry{raw mode}
{
  name={raw mode},
  description={A collection of behaviors differing from \gls{cooked mode},
   originating in the Version 7 terminal driver. Notably, the terminal driver no
   longer buffers input or generates signals. Raw mode can be entered with the
   \texttt{cfmakeraw()} termios call, where available}
}

\newglossaryentry{rgb}
{
  name={RGB},
  description={The Red, Green, Blue color model. In the context of Notcurses,
    this almost always means a 24-bit value, with 8 bits per component (some
    contexts accept a 32-bit value, where the least significant 8 bits are
    ignored)}
}

\longnewglossaryentry{terminal}
{
  name={terminal},
description={In the abstract, a means for entering data into a computer
  (typically involving a keyboard, and often a pointing device), and a means
  for displaying that conmputer's output. In the concrete, electromechanical
  devices for doing the same. I use ``terminal'' in the abstract sense
  throughout this book, but both meanings are in common use:

  \begin{denseitemize}
  \item I pull my chair up to my desk and \textit{engage my terminal}.
  \item I press Meta+F1 and \textit{bring up a terminal}.
  \end{denseitemize}

 Only the mad might refer to a physical device as a terminal emulator}
}

\longnewglossaryentry{terminalemulator}
{
  name={terminal emulator},
description={A program in a GUI environment providing a
  two-dimensional array of cells. A terminal emulator differs from a
  framebuffer/canvas in that each of these cells contains rendered (and
  possibly stylized) font glyphs, as opposed to a single pixel. Terminal
  emulator processes are connected to the slave ends of pseudoterminals.

 The pure character graphics environment of e.g. Linux or FreeBSD outside
  of X11 or Wayland is technically a terminal emulator, but almost always
  referred to as a ``console''. When I say ``terminal emulator'' in this book,
  I do not mean to include consoles}
}

\newglossaryentry{termcap}
{
  name={termcap},
description={A deprecated terminal capability database, superseded by \gls{terminfo}}
}

\newglossaryentry{terminfo}
{
  name={terminfo},
description={A database describing terminals and their capabilities. It is
  primarily indexed by terminal name, yielding an entry covering several
  hundred capabilities. These capabilities are then indexed by name. A
  terminal environment is responsible for setting the \texttt{TERM} environment
  variable to the correct primary terminfo index. termcap has been
  deprecated by the introduction of terminfo, which is superior in every way}
}

\newglossaryentry{termios}
{
  name={termios},
description={A POSIX API and data structure for working with arbitrary terminals.
  These functions all start with \texttt{tc} or \texttt{cf}, and control
  low-level behavior. See \texttt{termios(3)}\cite{termios}}
}

\newglossaryentry{ucs}
{
  name={UCS},
  description={See \Gls{isoiec10646}}
}

\newglossaryentry{ucs2}
{
  name={UCS-2},
description={A fixed-length encoding of the Basic Multilingual Plane. Each
  character is encoded to a single 16-bit code unit, for a total of 2 bytes
  per encoded character. UCS-2 cannot encode all of UCS, and is infrequently
  used. An amendment to \Gls{isoiec10646} extended UCS-2 to \Gls{utf16}}
}

\newglossaryentry{ucs4}
{
  name={UCS-4},
description={A fixed-length encoding of \Gls{ucs} to 32-bit code units.
  \Gls{utf32} is a proper subset of UCS-4\cite{rfc3629}}
}

\newglossaryentry{unicode}
{
  name={Unicode},
description={UCS as defined by \Gls{isoiec10646} (which Unicode inducts), plus
  rules on how to use that code set. Unicode is defined by the Unicode
  Consortium}
}

\newglossaryentry{utf1}
{
  name={UTF-1},
description={An early garbage form of UTF-8. \Gls{isoiec10646} Annex G, since deprecated.
  Unfortunately, the page might not yet have been ripped out at your local
  library. Don't use UTF-1},
}

\newglossaryentry{utf8}
{
  name={UTF-8},
description={``The problems outlined here go away when exclusively using UTF-8, which is one of the many reasons that is now the mandatory encoding for all things.''---WHATWG Encoding Standard\cite{whatwg}.
  The only Unicode encoding in which you ever ought pass information
  between tools\cite{utf8everywhere}. FIXME}
}

\newglossaryentry{utf16}
{
  name={UTF-16},
description={A variable-length encoding of UCS. Each character is encoded to
  either one or two 16-bit code units, for a total of 2 or 4 bytes per
  encoded character. For fun, it comes in two endiannesses. There's very little
  good about UTF-16, and it ought neither be purchased, nor accepted as a gift.
  Perhaps not coincidentally, it is the internal encoding of both Microsoft
  Windows and Java}
}

\newglossaryentry{utf32}
{
  name={UTF-32},
description={A fixed-length encoding of UCS. Each character is encoded to a
  single 32-bit code unit, for a total of 4 bytes per encoded character. UTF-32
  is a restricted subset of \Gls{ucs4}\cite{rfc3629}}
}

\newglossaryentry{virtualterm}
{
  name={virtual terminal},
description={Some consoles multiplex multiple instances. Each is
  typically referred to as a virtual terminal. I have never heard anyone
  refer to the multiple tabs of a terminal emulator as ``virtual terminals'',
  and would look at them strangely if I did}
}

\newglossaryentry{whiptail}
{
  name={whiptail},
description={An application for providing TUI modal dialogs, similar to \texttt{dialog}. It is written using Newt}
}

\newglossaryentry{widechar}
{
  name={wide character},
  description={This has at least three distinct meanings:
\begin{denseitemize}
\item \textbf{\texttt{wchar\_t}}: A data type in C (where it is an alias of an integral
  type) and \CC (where it is a fundamental type). \texttt{wchar\_t} by the
  Standard ought be ``large enough to represent any supported character code
  point''. The Redmond brain trust turned around and helpfully defined
  \texttt{wchar\_t} to be 16 bits on the UTF-16-blighted Microsoft Windows,
  where it cannot represent code points beyond the Basic Multilingual Plane.
  Use \texttt{char32\_t} if you need a type that can represent any Unicode
  point. Unfortunately, standard C interfaces such as \texttt{wcwidth(3)} want
  \texttt{wchar\_t}. A ``wide character'' in the C/\CC context means ``a character
  from a set which cannot be represented using the \texttt{char} type.'' Nothing is implied
  about the column width or the encoded byte length of a given character.
\item \textbf{Wide East Asian characters}: Defined in Unicode Standard Annex \#11.
    Every Unicode character has as its \textit{default width} property one of
    six values: \textit{Ambiguous}, \textit{Fullwidth} (全形/ 全角),
    \textit{Halfwidth} (半形/半角), \textit{Narrow}, \textit{Wide}, and \textit{Neutral}
    aka ``Not East Asian''. All six resolve to either ``narrow'' or ``wide'', with
    the resolution depending on context. All fullwidth characters are wide, and
    all halfwidth characters are narrow. Traditionally, ``narrow'' characters
    occupy half an \gls{em} in East Asian writing, but today it is more common
    for ``wide'' characters to occupy two ems.
\item \textbf{Multicolumn EGCs}: Depending on the font, an EGC might occupuy
  more than one column. The ANSI/ISO C function \texttt{wcwidth(3)} supposely
  returns the number of columns a \texttt{wchar\_t} will occupy, but it can't
  be trusted---this is ultimately a property of the font. Perhaps most
  spectacular is \texttt{U+FDFD ARABIC LIGATURE BISMILLAH AR-RAHMAN AR-RAHEEM}
  aka \begin{arab}[utf]﷽\end{arab}
\end{denseitemize}}
}

\newglossaryentry{em}
{
  name={em},
  description={A typographic term referring to the width of a glyph. It is
    equal to the current point size. It derives its name from the width of a
    capital M, which was often the widest glyph in a typeface.}
}

\newglossaryentry{sgr}
{
  name={SGR},
    description={Select Graphic Rendition. An ANSI escape (defined in
      \Gls{ecma48}) and implemented (to one degree or another) by the
      VT100\cite{vt100} (and most terminals since)\footnote{Support is quite varied
      among terminals. VT100 didn't even do color.}. 64 different
      attributes are defined by ECMA-48, including e.\ g.\ underline and
      eight colors (plus a default color). Sometimes the ``bold'' (1) and
      ``faint'' (2) attributes of SGR cause a different color to be selected;
      sometimes they change the glyph}
}

\newglossaryentry{cielab}
{
  name={CIELAB},
  description={A 1976 color space (also known as ``CIE L*a*b'' or ``Lab'') defined such that
  numerical changes correspond to roughly the same perceptible change. It is
  a more perceptually uniform derivative of 1931's CIEXYZ, influenced by the
  Munsell color system. CIELAB colors are not absolute, but defined relative
  to a CIEXYZ white point. The International Commission on Illumination further
  developed CIEXYZ and CIELAB into the 2002 CIECAM02 color appearance model}
}

\newglossaryentry{boundingbox}
{
  name={bounding box},
  description={The smallest rectilinear area necessary for containing some
    sprite or other displayed object},
}

\newglossaryentry{isoiec14755}
{
  name={ISO/IEC 14755:1996},
  description={``Information technology---Input methods to enter characters from the repertoire of ISO/IEC 10646 with a
    keyboard or other input devices''}
}
