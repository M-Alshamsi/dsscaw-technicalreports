\section{Using ncplanes}
\label{ncplane}
As mentioned in Chapter~\ref{sec:fullscreen}, \texttt{ncplane}s (henceforth
simply planes) are the fundamental drawing surface of Notcurses. A Notcurses
instance contains a z-axis on which planes are totally ordered\footnote{Future
releases of Notcurses might relax this to a partial ordering, allowing
multiple ncplanes to partition a logical level. See
\url{https://github.com/dankamongmen/notcurses/issues/184}.}. In addition, it
always contains at least one plane, the \textit{standard plane}. This plane's
origin is always defined to be the rendering area's origin. It is always
exactly as large as the rendering area, and it cannot be destroyed.

Chapter \ref{sec:notcursesfuncs} introduced one function that creates a new
plane: \texttt{ncplane\_new()}. In addition, there is \texttt{ncplane\_aligned()}.

\begin{listing}[!htbp]
\begin{minted}{C}
// Create a new ncplane at the specified 'yoff', of the specified 'rows' and 'cols'. Align this plane
// according to 'align' relative to 'n'.
struct ncplane* ncplane_aligned(struct ncplane* n, int rows, int cols, int yoff, ncalign_e align, void* opaque);
\end{minted}
\caption{Creating a new plane aligned relative to another.}
\end{listing}

A plane is defined by:
\begin{denseitemize}
\item{A packed ``framebuffer'' of \texttt{cell} structures. Cells are discussed
    in detail in~\ref{sec:cells}; it is enough now to know that each has an
    EGC, a set of attributes, and a fore- and background color.}
\item{A ``base cell'', rendered for any cell with a null EGC.}
\item{A cursor position relative to the plane's origin.}
\item{The plane's position relative to the visible area's origin.}
\item{A two-dimensional size.}
\item{An ``egcpool'' providing backing storage for the framebuffer.}
\item{An opaque pointer, controlled by the application.}
\item{A current set of attributes and colors.}
\item{A pointer to the plane below this one, or \texttt{NULL} for the bottommost plane.}
\end{denseitemize}

However a plane is created---including the standard plane---it is initialized
in the same way. All cells--both the base cell and those of the
framebuffer---are zeroed out. A zeroed cell has the null glyph (UTF-8
value ``00''), no attributes, and the default foreground and background color
(default colors are always opaque). Note that planes are thus by default
glyph-transparent but color-opaque. The cursor is placed at the origin. The
plane's current attributes and channels are likewise zeroed out. The plane is
pushed onto the top of the z-axis and assigned an initialized egcpool
(see~\ref{sec:egcpools}).

A plane can be duplicated with \texttt{ncplane\_dup()}. This will create a new
plane of the same geometry at the top of the z-axis. It will then have all
other properties duplicated, using its own egcpool.
\begin{listing}[!htbp]
\begin{minted}{C}
// Duplicate an existing ncplane. The new plane will have the same geometry, will duplicate all content, and
// will start with the same rendering state. The new plane will be immediately above the old one on the z axis.
struct ncplane* ncplane_dup(struct ncplane* n, void* opaque);
\end{minted}
\caption{Duplicating a plane.}
\end{listing}

Besides the default plane, planes may occupy any positive size (both the number
of rows and columns must be greater than zero), and have their origins any
integer offset from the visual origin. It is possible for a plane to be a
superset of the visual area, a subset, to exactly match the visual area, to
partially overlap it, or even to be entirely off-screen. A plane can be
moved to any coordinate, but the plane's cursor cannot be moved off the plane.

Any plane save the standard plane may be destroyed with \texttt{ncplane\_destroy()}.
All planes save the standard plane may be destroyed in one fell swoop with
\texttt{notcurses\_drop\_planes()}.
\begin{listing}[!htbp]
\begin{minted}{C}
// Destroy the specified ncplane. None of its contents will be visible after the next
// call to notcurses_render(). It is an error to attempt to destroy the standard plane.
int ncplane_destroy(struct ncplane* ncp);

// Destroy all ncplanes other than the standard plane.
void notcurses_drop_planes(struct notcurses* nc);
\end{minted}
\caption{Destroying planes.}
\end{listing}

\subsection{Moving and resizing planes}
Even the standard plane can be reordered along the z-axis. \texttt{ncplane\_move\_top()}
and \texttt{ncplane\_move\_bottom()} are absolute, moving the specified plane
to the top or bottom of the z-axis, respectively. \texttt{ncplane\_move\_above()}
and \texttt{ncplane\_move\_below()} are relative, moving the plane immediately
above or below another one. It is an error to try and move a plane below or above itself,
or above or below \texttt{NULL}. Likewise, an error will be returned if the relative
plane does not exist on the z-axis.
\begin{listing}[!htbp]
\begin{minted}{C}
// Splice ncplane 'n' out of the z-buffer, and reinsert it at the top or bottom.
int ncplane_move_top(struct ncplane* n);
int ncplane_move_bottom(struct ncplane* n);

// Splice ncplane 'n' out of the z-buffer, and reinsert it above 'above'.
int ncplane_move_above_unsafe(struct ncplane* restrict n, struct ncplane* restrict above);

static inline int
ncplane_move_above(struct ncplane* n, struct ncplane* above){
  if(n == above){
    return -1;
  }
  return ncplane_move_above_unsafe(n, above);
}

// Splice ncplane 'n' out of the z-buffer, and reinsert it below 'below'.
int ncplane_move_below_unsafe(struct ncplane* restrict n, struct ncplane* restrict below);

static inline int
ncplane_move_below(struct ncplane* n, struct ncplane* below){
  if(n == below){
    return -1;
  }
  return ncplane_move_below_unsafe(n, below);
}

// Return the plane above this one, or NULL if this is at the top.
struct ncplane* ncplane_below(struct ncplane* n);
\end{minted}
\caption{Moving planes on the z axis.}
\end{listing}

All planes other than the standard plane can be moved in the x- and y-dimensions.
\begin{listing}[!htbp]
\begin{minted}{C}
// Move this plane relative to the standard plane. It is an error to attempt to // move the standard plane.
int ncplane_move_yx(struct ncplane* n, int y, int x);
\end{minted}
\caption{Moving planes on the x and y axis.}
\end{listing}

\textbf{FIXME ncplane\_resize()...}

Sometimes it is useful to translate coordinates between planes, or between the
visible area and planes (this latter is particularly useful when interpreting
mouse clicks; see Chapter~\ref{sec:input}).
\begin{listing}[!htbp]
\begin{minted}{C}
// provided a coordinate relative to the origin of 'src', map it to the same absolute coordinate
// relative to thte origin of 'dst'. either or both of 'y' and 'x' may be NULL.
void ncplane_translate(const struct ncplane* src, const struct ncplane* dst, int* restrict y, int* restrict x);

// Fed absolute 'y'/'x' coordinates, determine whether that coordinate is within the ncplane 'n'. If not, return false.
// If so, return true. Either way, translate the absolute coordinates relative to 'n'. If the point is not within 'n',
// these coordinates will not be within the dimensions of the plane.
bool ncplane_translate_abs(const struct ncplane* n, int* restrict y, int* restrict x);
\end{minted}
\caption{Translating coordinates between planes.}
\end{listing}

\subsection{Cells}
\label{sec:cells}
Each coordinate of an plane corresponds to a \texttt{cell}. The cell definition
is exposed to the application, though it should not generally be directly
manipulated. A multicolumn cell (a cell containing an EGC of $n$ columns where
$n>1$) overrides the $n-1$ following cells. Since there are always a fixed
number of cells, this means that the overridden cells are skipped during
rendering, as well as being zeroed out at the time the multicolumn EGC is
written to the cell.
\begin{listing}[!htbp]
\begin{minted}{C}
typedef struct cell {
  // These 32 bits are either a single-byte, single-character grapheme cluster (values 0–0x7f), or
  // an offset into a per-ncplane attached pool of varying-length UTF-8 grapheme clusters.
  uint32_t gcluster;          // 4B -> 4B
  // NCSTYLE_* attributes (16 bits) + 8 foreground palette index bits + 8 background palette index
  // bits. palette index bits are used only if the corresponding default color bit *is not* set,
  // and the corresponding palette index bit *is* set.
  uint32_t attrword;          // + 4B -> 8B
  // (channels & 0x8000000000000000ull): left half of wide character
  // (channels & 0x4000000000000000ull): foreground is *not* "default color"
  // (channels & 0x3000000000000000ull): foreground alpha (2 bits)
  // (channels & 0x0800000000000000ull): foreground uses palette index
  // (channels & 0x0700000000000000ull): reserved, must be 0
  // (channels & 0x00ffffff00000000ull): foreground in 3x8 RGB (rrggbb)
  // (channels & 0x0000000080000000ull): right half of wide character
  // (channels & 0x0000000040000000ull): background is *not* "default color"
  // (channels & 0x0000000030000000ull): background alpha (2 bits)
  // (channels & 0x0000000008000000ull): background uses palette index
  // (channels & 0x0000000007000000ull): reserved, must be 0
  // (channels & 0x0000000000ffffffull): background in 3x8 RGB (rrggbb)
  uint64_t channels;          // + 8B == 16B
} cell;
\end{minted}
\caption{The \texttt{cell} definition.}
\end{listing}
The \texttt{gcluster} field is a 32-bit number. If the value is less than 128,
it directly specifies its UTF-8 encoded character. Since Unicode's first 128
values are taken directly from ASCII, this means the entirety of ASCII can be
represented in-line. If the value is greater than or equal to 128, it is a
bias-128 index into the plane's associated egcpool. Since egcpools are per-plane,
this implies that it is unsafe to blindly copy a cell from one plane to another.

Applications generally need not work directly with cells, though sometimes it
is easiest to do so. The usual reason for working with a cell is either to set
all three properies of output at once (glyph, attributes, and colors), or to
receive all three properties at once when retrieving a coordinate's data.

\subsection{egcpools}
\label{sec:egcpools}

\subsection{Alpha blending and plane transparency}
