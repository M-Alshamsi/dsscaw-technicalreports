\section{Collecting and dispatching input}
\label{sec:input}
\epigraph{The X server primarily handles two kinds of hardware: input devices and output devices. Surprisingly, the input handling tends to be the more difficult and complicated of the two. Input is multi-source, concurrent, and highly dependent on complex user preferences.}{The New X Developer's Guide\cite{x}}
To enter arbitrary Unicode using the keyboard, try pressing Ctrl+Shift+u. If
successful, you ought see an underlined or otherwise stylized lowercase 'u'.
The Unicode code point can be entered (each number will show up as you type it),
followed by Enter. The sequence will then reduce to an EGC.

The most fundamental call is \texttt{notcurses\_getc()}. This can operate as a
non-blocking, timed, or blocking call. Provide a \texttt{NULL}\texttt{struct timespec} '\texttt{ts}'.
\texttt{notcurses\_getc()} to block until input is received, or the call is
interrupted by a signal (prepare the \texttt{sigset\_t} parameter to mask signals
as necessary). Provide the desired timeout in '\texttt{ts}' for a timed call,
or zero out '\texttt{ts}' for a pure nonblocking call. On timeout, 0 is returned.
On an error, -1 is returned. Otherwise, a \texttt{char32\_t} is returned carrying
a single UTF-32 Unicode codepoint. If the \texttt{ncinput} parameter '\texttt{ni}'
is not \texttt{NULL}, it will be filled in with the codepoint, any applicable
keyboard modifiers, and the cell of the input\footnote{Coordinates are currently
reported only for pointing devices, not keyboards.}. Two helpers exist to simplify
standard use cases: \texttt{notcurses\_getc\_nblock()} and \texttt{notcurses\_getc\_blocking()}
do exactly what you'd expect.

It's important to note the difference in granularity of display and input. For
display, we must write a UTF-8 encoded EGC as a single unit. Repeated calls do
not combine within a cell. On input, however, we use UTF-32 and read a single
code point, \textit{not} an entire EGC. This is done to simplify input
processing. It is recommended that, should you find yourself needing to
implement backspace functionality, you implement it at the EGC level.

Notcurses places the terminal into cbreak mode (recall Chapter~\ref{section:tty}).
This means that input is made immediately available, without waiting for a newline.
The translation of certain key combinations (as specified by \texttt{termios})
into signals by the kernel line discipline still occurs\textellipsis for now.
It is likely that Notcurses will move to raw mode by the 2.0.0 release, in
which case it will have to handle signal translation itself\footnote{The purpose
of this planned change is to reliably acquire the state of keyboard modifiers---Shift,
Ctrl, Alt, and their ilk. Currently, Notcurses reliably reports these modifiers
only for mouse events. Furthermore, it is not currently possible to differentiate
e.g. Ctrl+I from TAB (recall Table~\ref{table:c0maps}), but it will be once
Notcurses begins using raw mode.}.

\begin{listing}[!htb]
\begin{minted}{C}
// See ppoll(2) for more detail. Provide a NULL 'ts' to block at length, a 'ts' of 0 for non-blocking operation,
// and otherwise a timespec to bound blocking. Signals in sigmask (less several we handle internally) will be
// atomically masked and unmasked per ppoll(2). It should generally contain all signals. Returns a single Unicode
// code point, or (char32_t)-1 on error. 'sigmask' may be NULL. Returns 0 on a timeout. If an event is processed,
// the return value is the 'id' field from that event. 'ni' may be NULL.
char32_t notcurses_getc(struct notcurses* n, const struct timespec* ts, sigset_t* sigmask, ncinput* ni);

// 'ni' may be NULL if the caller is uninterested in event details. If no event is ready, returns 0.
static inline char32_t notcurses_getc_nblock(struct notcurses* n, ncinput* ni){
  sigset_t sigmask;
  sigfillset(&sigmask);
  struct timespec ts = { .tv_sec = 0, .tv_nsec = 0 };
  return notcurses_getc(n, &ts, &sigmask, ni);
}

// 'ni' may be NULL if the caller is uninterested in event details.
// Blocks until an event is processed or a signal is received.
static inline char32_t notcurses_getc_blocking(struct notcurses* n, ncinput* ni){
  sigset_t sigmask;
  sigemptyset(&sigmask);
  return notcurses_getc(n, NULL, &sigmask, ni);
}
\end{minted}
\caption{Input can be acquired in nonblocking, blocking, or timed fashion.}
\label{listing:input}
\end{listing}

Mouse events are reported only after a successful call to
\texttt{notcurses\_mouse\_enable()}, and will no longer be reported following
\texttt{notcurses\_mouse\_disable()}. Even when enabled, events are only
returned while a button is held. There will be one event for the initial button
press, one event for each cell into which the mouse moves while holding down
the button, and one when the button is released. See Listing~\ref{listing:mice}.

\begin{listing}[!htb]
\begin{minted}{C}
// Enable the mouse in "button-event tracking" mode with focus detection and
// UTF8-style extended coordinates. On failure, -1 is returned. On success, 0
// is returned, and mouse events will be published to notcurses_getc().
int notcurses_mouse_enable(struct notcurses* n);

// Disable mouse events. Any events in the input queue can still be delivered.
int notcurses_mouse_disable(struct notcurses* n);
\end{minted}
\caption{Mouse events must be explicitly enabled, and can be disabled.}
\label{listing:mice}
\end{listing}

Input functions may be called concurrently with any output or read-only
functions, but only one thread at a time may call into the input layer via any
of its entry points.

There are many keys on a typical PC-104 keyboard which do not correspond to any
Unicode code point. Examples include function keys, arrow keys, all the mouse
buttons, and PageUp/PageDown\footnote{Though there \textit{are} code points for
Page Up (\texttt{⇞}) and Down(\texttt{⇟}), I've never seen them emitted by the keyboard driver.}.
For this purpose, Notcurses takes a chunk of Unicode's Private Supplementary Area-B,
a plane explicitly left empty for use with local definitions. The lengthy set
of \texttt{NCKEY\_} macros are exported to allow trivial matching against
these ``synthesized characters''. Finally, \texttt{NCKEY\_RESIZE} is generated
and pushed into the input queue whenever \texttt{SIGWINCH} is handled.

Input is not bound to any particular plane, and is indeed processed only minimally
by Notcurses. The input queue is associated with a \texttt{notcurses} context
as a whole. In particular, UI widgets do not automatically handle ``their''
input. Creating an \texttt{ncmenu} will see the menu drawn, but clicking it
won't take any action by itself. Instead, each widget type provides a \texttt{*\_handle\_input()}
function. Typically, your program should get the input, check it against any
input handling your program performs, and invoke \texttt{*\_handle\_input()}
appropriately for each UI widget active. Perform these checks in whatever order
is appropriate for your program's concept of ``focus''. If the widget's input
handler returns true, consider the input ``consumed'' by that widget.

Just because a widget \textit{consumes} input doesn't mean it \textit{acts upon}
that input. Essentially, if the input results in actions limited to the widget,
those actions are performed; if the input would result in control flow external
to the widget, they are not. I know, I know. For now (as of Notcurses 1.2.3):
\begin{denseitemize}
\item{Selector and multiselect widgets effect scrolling (at both the line and
      page level) themselves, using keyboard or mouse. Multiselect effects toggling
      itself, using keyboard or mouse. You must destroy the widget, and get its
      current state before doing so.}
\item{Menu effects all movement and shading/unshading, using keyboard or mouse.
      You must get its current state when an item is executed.}
\item{There does not yet exist a \texttt{reel\_handle\_input()} function. You
      must manipulate reels directly.}
\end{denseitemize}
If you think this an unsatisfactory state of affairs, you're not alone. Expect
changes to the input system no later than Notcurses 2.0.0.
