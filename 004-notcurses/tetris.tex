\subsection{Example: let's rip off tetris}
\label{sec:casestudy}
Recall our exploration of tetriminos from Chapter~\ref{sec:simpleloop}. We
know enough now to turn this into an actual game of console Tetris. Our
implementation is fewer than 200 lines total, yet it unites most of the
techniques introduced in the past few chapters. For fun, we'll do this in \CC,
using Marek Habersack's \CC wrappers (installed with Notcurses).

Our \texttt{main()} will parse command line options, set up a \texttt{Tetris}
object appropriately, and watch for input. The \texttt{Tetris} object provides
the necessary interface to the input loop:

\begin{denseitemize}
\item{\texttt{MoveLeft()} and \texttt{MoveRight()}},
\item{\texttt{MoveDown()}},
\item{\texttt{Pause()}}, and
\item{\texttt{RotateCCw} and \texttt{RotateCw}}
\end{denseitemize}

The general structure of our solution is thus:

\begin{denseitemize}
\item{A background is drawn onto the standard plane.}
\item{A new plane is created for the game area. Why make a new plane? Recall
    the beginning of Chapter~\ref{sec:planes}: \textit{we use a plane wherever we
    benefit from distinct state}. By keeping the game area on its own plane,
    we can trivially move it in response to terminal resizes, and likewise
    trivially test whether the current piece can move in a given direction.}
\item{Two tetrimino planes are created, one for the current piece, and one for
    the next piece. The current piece descends from the top of the game area.
    Upon reaching its final position, its plane is added to the game plane
    using \texttt{ncplane\_mergedown()}. The next piece is brought to the top
    of the game area, and the current piece is redrawn on its way to becoming
    the next piece.}
\item{We need no external state save the score and the order of pieces. The order
    of pieces is left up to the PRNG. We only need track the score and
    our four planes. The validity of a given movement can be checked entirely
    by reflection, using \texttt{ncplane\_at\_yx()}.}
\end{denseitemize}

\textbf{FIXME FIXME FIXME}
