\section{Terminal mechanics}
\label{section:tty}
\epigraph{The tty layer is one of the very few pieces of kernel code that scares the hell out of me.}{Ingo Molnar\footnote{When Ingo Molnar is scared, we all ought be scared.}\cite{molnarhell}}
You won't often need to deal with the gritty details of terminal access and
manipulation. It's still important to understand what's going on behind the
abstraction, especially for when things go wrong. As was made clear in
Chapters~\ref{sec:direct} and~\ref{sec:fullscreen}, Notcurses requires a
proper terminal definition and a handle to a terminal device, or initialization
will fail (NCURSES requires the same). With that said, the UNIX terminal layers
have never been, and are not now for the faint of heart. Extending back to the
AT\&T dark ages, they are configured primarily through messy \texttt{ioctl}s
and the slightly-less-messy \texttt{termios} API.

Modern workstations support a variety of physical and virtual terminal devices:
\begin{denseitemize}
\item{Honest-to-Bog serial terminals, probably using the RS-232\cite{rs232}
      protocol over a D-subminiature 25-pin (DB-25M) or 9-pin (DE-9M)
      connector (see Table~\ref{table:serial}).}
\item{Virtual consoles on text-based video, plus a keyboard.}
\item{Virtual framebuffer consoles on graphics-based video, plus a keyboard.} 
\item{Terminal emulators in a graphical environment, plus brokered input devices.}
\item{Pseudoterminals hooked up to network connections.}
\end{denseitemize}

\begin{table}
  \centering
  \begin{tabular}{ |c|c|c|c| }
    \hline
    Signal & DB-25M & DE-9M & Originator \\
    \hline
    \hline
    Protective ground & 1 & & x \\
    \hline
    Transmitted data & 2 & 3 & DTE \\
    \hline
    Received data & 3 & 2 & DCE \\
    \hline
    Request to send & 4 & 7 & DTE \\
    \hline
    Clear to send & 5 & 8 & DCE \\
    \hline
    Data set ready & 6 & 6 & DCE \\
    \hline
    Signal ground & 7 & 5 & x \\
    \hline
    Carrier detect & 8 & 1 & DCE \\
    \hline
    Data terminal ready & 20 & 4 & DTE \\
    \hline
    Ring indicator & 22 & 9 & DCE \\
    \hline
  \end{tabular}
  \caption[RS-232/EIA-232 pin mappings]{RS-232/EIA-232 D-subminiature pins\\
    (DCE=Data circuit equipment, DTE=Data terminal equipment)}
  \label{table:serial}
\end{table}

\textbf{FIXME FIXME acquisition of a tty (getty->login, ssh->pty),
  internals of kernel tty/pty devices, session groups, signal distribution,
  \texttt{/dev/tty} and \texttt{/dev/ttyXX}s}
\textbf{FIXME FIXME other stuff}
\subsection{Escape codes ANSI and otherwise}
\label{sec:escapes}
My diagrams are adapted in part from those of Linus Akesson\cite{ttydemystified}.
\textbf{FIXME diagrams!}

In a relic from teletypes, ISO 646, ISO 2022, and ECMA-35 described use of
non-destructive backspace to produce composed characters from spacing ones.
This had been eliminated by ISO 4873, ECMA-43, and ISO 8859.

\textbf{FIXME FIXME other stuff}
