\section{Writing and styling text}
\label{sec:output}
The most fundamental aspect of textual interfaces is, after all, text. All
valid UTF-8 can be written to a plane. If scrolling has been enabled for a
plane, any amount of text can be written. If scrolling has not been enabled,
output can only be generated through the end of the current line. All text
output functions return the number of screen columns written as their primary
return value (or a negative number on failure), and load the number of bytes
written into an auxiliary value-result parameter. If the supplied output would
exceed the line, a short number of columns will be returned. Scrolling is
disabled by default on the standard plane, and on all new planes.

\begin{listing}[!htb]
\begin{minted}{C}
// Move the cursor to the specified position (the cursor needn't be visible).
// Returns -1 on error, including negative parameters, or ones exceeding the
// plane's dimensions.
int ncplane_cursor_move_yx(struct ncplane* n, int y, int x);

// Get the current position of the cursor within n. y and/or x may be NULL.
void ncplane_cursor_yx(const struct ncplane* n, int* restrict y, int* restrict x);
\end{minted}
\caption{Cursor management. Each plane has its own cursor.}
\label{list:cursor}
\end{listing}

The cursor is advanced appropriately to the cell just beyond the output. If the
entire line was written, and scrolling is not enabled, the cursor will be
off-plane and must be repositioned before writing any further. If scrolling is
enabled, the cursor will either move to the first column of the next line, or
if the cursor is on the last line, the plane will be scrolled by one line (and
the cursor will move to the first column). If there is an error following some
output, the cursor will be positioned following the generated output. Comparing
the new cursor location to the old can thus reveal the amount of output generated
in the event of a failure.

To check that there was no failure, then, verify that the return value is not
negative. The entirety of the input was written if any of the following is
true:

\begin{denseitemize}
\item{The return value is equal to the number of columns required to represent the input.
    \texttt{mbswidth()}(Listing ~\ref{list:mbswidth}) can be used to find the number of columns required by
    the input.}
\item{The number of bytes written is equal to the number of bytes supplied. Since Notcurses
   supports only ASCII, \texttt{strlen()} can be used to find the number of
   bytes supplied.}
\item{The cursor is not beyond the end of the plane, and scrolling is disabled
 on the plane.}
\end{denseitemize}

\begin{listing}[!htb]
\begin{minted}{C}
// Calculate the size in columns of the provided UTF8 multibyte string.
int mbswidth(const char* mbs);
\end{minted}
\caption{\texttt{mbswidth()} counts columns in a multibye string.}
\label{list:mbswidth}
\end{listing}

\subsection{Writing text to planes}
\label{sec:outputtext}

Multiple families of functions are available for writing to planes. Only valid
encoded sequences from the active locale's encoding can be output. Attempting
to e.g. write UTF-8 characters while using the \texttt{ANSI\_X3.4-1968} encoding
will fail as soon as a non-ASCII character is submitted.

The base form of each family places its output at the plane's current cursor
location. Each family has a \texttt{\_xy()}-suffixed form which moves the
cursor as specified prior to beginning output. Supplying -1 to the \texttt{x}
and \texttt{y} parameters of these forms doesn't move the cursor on the
relevant axis. Supplying -1 to both decays to the base function of the family.
The \texttt{\_stainable()}-suffixed form updates the glyphs of a plane without
changing the attributes or channels.

\begin{listing}[!htb]
\begin{minted}{C}
// Replace the cell at the specified coordinates with the provided cell 'c',
// and advance the cursor by the width of the cell (but not past the end of the
// plane). On success, returns the number of columns the cursor was advanced.
// On failure, -1 is returned.
int ncplane_putc_yx(struct ncplane* n, int y, int x, const cell* c);

// Call ncplane_putc_yx() for the current cursor location.
static inline int
ncplane_putc(struct ncplane* n, const cell* c){
  return ncplane_putc_yx(n, -1, -1, c);
}
\end{minted}
\caption{Output of \texttt{cell}s to planes.}
\label{list:putcell}
\end{listing}

ASCII (and thus the lowest 128 UTF-8 encoded characters) can be written directly
with the \texttt{putsimple} family. Note that any control character will be replaced with
a space.

\begin{listing}[!htb]
\begin{minted}{C}
// Replace the EGC underneath us, but retain the styling. The current styling
// of the plane will not be changed.
//
// Replace the cell at the specified coordinates with the provided 7-bit char
// 'c'. Advance the cursor by 1. On success, returns 1. On failure, returns -1.
// This works whether the underlying char is signed or unsigned.
int ncplane_putsimple_yx(struct ncplane* n, int y, int x, char c);

// Call ncplane_putsimple_yx() at the current cursor location.
static inline int
ncplane_putsimple(struct ncplane* n, char c){
  return ncplane_putsimple_yx(n, -1, -1, c);
}

// Replace the EGC underneath us, but retain the styling. The current styling
// of the plane will not be changed.
int ncplane_putsimple_stainable(struct ncplane* n, char c);
\end{minted}
\caption{Direct output of single-byte UTF-8 to planes.}
\label{list:putc}
\end{listing}

On systems where \texttt{wchar\_t} is at least twenty-five bits\footnote{\texttt{wchar\_t}
  is only 16 bits on some systems, usually those same brain-damaged ones which
  make great use of brain-damaged UTF-16. Such a \texttt{wchar\_t} can only
represent characters from the Basic Multilingual Plane; the other sixteen planes
require use of \textit{surrogate characters}.}, a single \texttt{wchar\_t} can
represent any UCS code point (though not necessarily any EGC). The
\texttt{putwc} family (Listing~\ref{list:putwc}) allows direct output of a single \texttt{wchar\_t}.

\begin{listing}[!htb]
\begin{minted}{C}
// Replace the cell at the specified coordinates with the provided wide char
// 'w'. Advance the cursor by the character's width as reported by wcwidth().
// On success, returns 1. On failure, returns -1.
static inline int ncplane_putwc_yx(struct ncplane* n, int y, int x, wchar_t w){
  wchar_t warr[2] = { w, L'\0' };
  return ncplane_putwstr_yx(n, y, x, warr);
}

// Call ncplane_putwc() at the current cursor position.
static inline int ncplane_putwc(struct ncplane* n, wchar_t w){
  return ncplane_putwc_yx(n, -1, -1, w);
}
\end{minted}
\caption{Direct output of a single \texttt{wchar\_t}.}
\label{list:putwc}
\end{listing}

EGCs can be output one at a time. Supplying multiple EGCs in a single buffer
to the \texttt{putegc} family (Listing~\ref{list:putegc}) will only ever see the first one output.

\begin{listing}[!htb]
\begin{minted}{C}
// Replace the cell at the specified coordinates with the provided EGC, and
// advance the cursor by the width of the cluster (but not past the end of the
// plane). On success, returns the number of columns the cursor was advanced.
// On failure, -1 is returned. The number of bytes converted from gclust is
// written to 'sbytes' if non-NULL.
int ncplane_putegc_yx(struct ncplane* n, int y, int x, const char* gclust, int* sbytes);

// Call ncplane_putegc() at the current cursor location.
static inline int
ncplane_putegc(struct ncplane* n, const char* gclust, int* sbytes){
  return ncplane_putegc_yx(n, -1, -1, gclust, sbytes);
}

// Replace the EGC underneath us, but retain the styling. The current styling
// of the plane will not be changed.
int ncplane_putegc_stainable(struct ncplane* n, const char* gclust, int* sbytes);
\end{minted}
\caption{Output of single EGCs to planes.}
\label{list:putegc}
\end{listing}

\begin{listing}[!htb]
\begin{minted}{C}
#define WCHAR_MAX_UTF8BYTES 6

// ncplane_putegc(), but following a conversion from wchar_t to UTF-8 multibyte.
static inline int
ncplane_putwegc(struct ncplane* n, const wchar_t* gclust, int* sbytes){
  // maximum of six UTF8-encoded bytes per wchar_t
  const size_t mbytes = (wcslen(gclust) * WCHAR_MAX_UTF8BYTES) + 1;
  char* mbstr = (char*)malloc(mbytes); // need cast for c++ callers
  if(mbstr == NULL){
    return -1;
  }
  size_t s = wcstombs(mbstr, gclust, mbytes);
  if(s == (size_t)-1){
    free(mbstr);
    return -1;
  }
  int ret = ncplane_putegc(n, mbstr, sbytes);
  free(mbstr);
  return ret;
}

// Call ncplane_putwegc() after successfully moving to y, x.
static inline int
ncplane_putwegc_yx(struct ncplane* n, int y, int x, const wchar_t* gclust,
                   int* sbytes){
  if(ncplane_cursor_move_yx(n, y, x)){
    return -1;
  }
  return ncplane_putwegc(n, gclust, sbytes);
}

// Replace the EGC underneath us, but retain the styling. The current styling
// of the plane will not be changed.
int ncplane_putwegc_stainable(struct ncplane* n, const wchar_t* gclust, int* sbytes);
\end{minted}
\caption{Output of single \texttt{wchar\_t}-encoded EGCs to planes.}
\label{list:putwegc}
\end{listing}

\begin{listing}[!htb]
\begin{minted}{C}
// Write a series of EGCs to the current location, using the current style.
// They will be interpreted as a series of columns (according to the definition
// of ncplane_putc()). Advances the cursor by some positive number of cells
// (though not beyond the end of the plane); this number is returned on success.
// On error, a non-positive number is returned, indicating the number of cells
// which were written before the error.
int ncplane_putstr_yx(struct ncplane* n, int y, int x, const char* gclustarr);

static inline int
ncplane_putstr(struct ncplane* n, const char* gclustarr){
  return ncplane_putstr_yx(n, -1, -1, gclustarr);
}

int ncplane_putstr_aligned(struct ncplane* n, int y, ncalign_e align, const char* s);
\end{minted}
\caption{Output of strings to planes.}
\label{list:putstr}
\end{listing}

\begin{listing}[!htb]
\begin{minted}{C}
// ncplane_putstr(), but following a conversion from wchar_t to UTF-8 multibyte.
static inline int ncplane_putwstr_yx(struct ncplane* n, int y, int x, const wchar_t* gclustarr){
  // maximum of six UTF8-encoded bytes per wchar_t
  const size_t mbytes = (wcslen(gclustarr) * WCHAR_MAX_UTF8BYTES) + 1;
  char* mbstr = (char*)malloc(mbytes); // need cast for c++ callers
  if(mbstr == NULL){
    return -1;
  }
  size_t s = wcstombs(mbstr, gclustarr, mbytes);
  if(s == (size_t)-1){
    free(mbstr);
    return -1;
  }
  int ret = ncplane_putstr_yx(n, y, x, mbstr);
  free(mbstr);
  return ret;
}

static inline int ncplane_putwstr_aligned(struct ncplane* n, int y, ncalign_e align, const wchar_t* gclustarr){
  int width = wcswidth(gclustarr, INT_MAX);
  int xpos = ncplane_align(n, align, width);
  return ncplane_putwstr_yx(n, y, xpos, gclustarr);
}

static inline int ncplane_putwstr(struct ncplane* n, const wchar_t* gclustarr){
  return ncplane_putwstr_yx(n, -1, -1, gclustarr);
}
\end{minted}
\caption{Output of wide strings to planes.}
\label{list:wputstr}
\end{listing}

Finally, analogues of \texttt{printf(3)} and \texttt{vprintf(3)} are provided (Listing~\ref{list:printf}).

\begin{listing}[!htb]
\begin{minted}{C}
// The ncplane equivalents of printf(3) and vprintf(3).
int ncplane_vprintf_aligned(struct ncplane* n, int y, ncalign_e align, const char* format, va_list ap);

int ncplane_vprintf_yx(struct ncplane* n, int y, int x, const char* format, va_list ap);

static inline int ncplane_vprintf(struct ncplane* n, const char* format, va_list ap){
  return ncplane_vprintf_yx(n, -1, -1, format, ap);
}

static inline int ncplane_printf(struct ncplane* n, const char* format, ...)
  __attribute__ ((format (printf, 2, 3)));

static inline int ncplane_printf(struct ncplane* n, const char* format, ...){
  va_list va;
  va_start(va, format);
  int ret = ncplane_vprintf(n, format, va);
  va_end(va);
  return ret;
}

static inline int ncplane_printf_yx(struct ncplane* n, int y, int x, const char* format, ...)
  __attribute__ ((format (printf, 4, 5)));

static inline int ncplane_printf_yx(struct ncplane* n, int y, int x, const char* format, ...){
  va_list va;
  va_start(va, format);
  int ret = ncplane_vprintf_yx(n, y, x, format, va);
  va_end(va);
  return ret;
}

static inline int ncplane_printf_aligned(struct ncplane* n, int y, ncalign_e align, const char* format, ...)
  __attribute__ ((format (printf, 4, 5)));

static inline int ncplane_printf_aligned(struct ncplane* n, int y, ncalign_e align, const char* format, ...){
  va_list va;
  va_start(va, format);
  int ret = ncplane_vprintf_aligned(n, y, align, format, va);
  va_end(va);
  return ret;
}
\end{minted}
\caption{Formatted output to planes.}
\label{list:printf}
\end{listing}

\subsection{The 32-bit \texttt{attribute} value}
\label{sec:attribute}
Each cell has a 32-bit \texttt{attribute} field, initialized to 0. Many functions
accept a 32-bit attribute, either to directly set a cell, or to apply to a number
of cells. This type (sometimes called \texttt{attrword} is logically a 2-byte
bitmask of \texttt{NCSTYLE\_} flags (see Table~\ref{table:styles}), followed by two 8-bit palette indices. The
palette indices may take any value from 0--255, but are used if and only if the
corresponding ``not default color'' \textit{and} ``palette-indexed color'' bits
are set in the channels (see below). Changing a channel to indicate default color
will \textit{not} reset the palette byte, but \textit{will} supercede
it\footnote{Applications are strongly encouraged not to treat these two bytes as free-use scratchpad, in
case these semantics change in the future. With that said, sometimes you've gotta
take sixteen bits wherever you can find them, especially when they represent 12.5%%
of your framebuffer.}.

\begin{table}[!htb]
  \centering
  \begin{tabular}{|l|l|l|}
    \hline
    Constant & Value & Comments \\
    \hline
    \hline
\texttt{NCSTYLE\_STANDOUT} &  0x00800000ul & Implementation-defined \\
\texttt{NCSTYLE\_UNDERLINE} & 0x00400000ul & \\
\texttt{NCSTYLE\_REVERSE} &   0x00200000ul & \\
\texttt{NCSTYLE\_BLINK} &     0x00100000ul & \\
\texttt{NCSTYLE\_DIM} &       0x00080000ul & Sometimes just a different color \\
\texttt{NCSTYLE\_BOLD} &      0x00040000ul & Sometimes just a different color \\
\texttt{NCSTYLE\_INVIS} &     0x00020000ul & \\
\texttt{NCSTYLE\_PROTECT} &   0x00010000ul & \\
\texttt{NCSTYLE\_ITALIC} &    0x01000000ul & Not commonly supported \\
    \hline
  \end{tabular}
  \caption{The \texttt{NCSTYLE} bits.}
  \label{table:styles}
\end{table}

It is not guaranteed that all styles are usable on a given terminal emulator.
As noted in Chapter~\ref{sec:capabilities}, the \texttt{notcurses\_supported\_styles()}
function can be used at runtime to determine which are available. 

Bold mode is the culprit responsible for the widespread misconception that there
are 16 or even 256 ANSI colors. Only 8 colors are defined in
ECMA-48\footnote{All 8 colors are defined for both fore- and background.}:
black, red, green, yellow, blue, magenta, cyan, and white. Some terminals
implemented bold and/or dim as additional colors, and these colors were often
available for direct selection. A terminal can faithfully implement all of ANSI
with only eight colors. \texttt{aixterm} supported 16 colors, which were
picked up by xterm and NCURSES. \texttt{rxvt-unicode} introduced 88 color support, which
was later picked up by \texttt{rxvt} and \texttt{xterm}\footnote{Why 88? Ignoring
possible connections to Neo-Nazis or \textit{Back to the Future}...I'm not quite
sure. The canonical 88-color palette is the 16 \texttt{aixterm} colors, a 4x4x4
color cube, and an 8-element grey ramp. Fair enough. But six bits can only represent
64 colors, and the seven necessary for 88 can represent 128. So why 88? Just to
mirror the 16+6x6x6+16+24 of 256 colors? Perhaps the other space in a byte is
used to encode additional information? I have been unable to reach a conclusive
answer. Oh well. According to my calculations\textellipsis when this baby hits 88, you're gonna see some serious shit!}.

\subsection{The 64-bit \texttt{channels} value}
\label{sec:channels}
In addition to the 32-bit \texttt{gcluster} and \texttt{attribute} fields, each
\texttt{cell} expends a further 64 bits (half of its 16 byte total) on the
\texttt{channels} field. Like \texttt{attribute} above, \texttt{channels} will
be used as both an identifier and a type. Furthermore, a 64-bit \texttt{channels}
variable is logically composed of two 32-bit \texttt{channel} values. The upper
32 bits describe the foreground channel, and the lower 32 bits describe the
background channel. The full eight bytes are broken down in Listing~\ref{list:channelbits}.

\begin{listing}[!htb]
\begin{minted}{C}
  // (channels & 0x8000000000000000ull): first column of wide EGC
  // (channels & 0x4000000000000000ull): foreground is *not* "default color"
  // (channels & 0x3000000000000000ull): foreground alpha (2 bits)
  // (channels & 0x0800000000000000ull): foreground uses palette index
  // (channels & 0x0700000000000000ull): reserved, must be 0
  // (channels & 0x00ffffff00000000ull): foreground in 3x8 RGB (rrggbb)
  // (channels & 0x0000000080000000ull): secondary column of wide EGC
  // (channels & 0x0000000040000000ull): background is *not* "default color"
  // (channels & 0x0000000030000000ull): background alpha (2 bits)
  // (channels & 0x0000000008000000ull): background uses palette index
  // (channels & 0x0000000007000000ull): reserved, must be 0
  // (channels & 0x0000000000ffffffull): background in 3x8 RGB (rrggbb)
\end{minted}
\caption{Bits of the \texttt{channels} type}.
\label{list:channelbits}
\end{listing}

The first bit of either channel is high if and only if the cell is part of a
multicolumn EGC When a cell is loaded via e.g. \texttt{cell\_load()}, the most
significant bit (63) of \texttt{channels} is set if and only if the EGC is wide. No
cell having a non-zero \texttt{gcluster} has the ``secondary column of wide EGC''
bit (31) set. This is set only in framebuffer cells having a null glyph due to
being ``covered'' by a preceding multicolumn EGC. These two bits should never
be manipulated by the user; Notcurses automatically manages them, and relies upon
them internally. You can of course read them, should you be so inclined. Three
functions are supplied for testing these bits (Listing~\ref{list:widetests}).

\begin{listing}[!htb]
\begin{minted}{C}
// Does the cell contain an East Asian Wide codepoint?
static inline bool cell_double_wide_p(const cell* c){
  return (c->channels & CELL_WIDEASIAN_MASK);
}

// Is this the right half of a wide character?
static inline bool cell_wide_right_p(const cell* c){
  return cell_double_wide_p(c) && c->gcluster == 0;
}

// Is this the left half of a wide character?
static inline bool cell_wide_left_p(const cell* c){
  return cell_double_wide_p(c) && c->gcluster;
}
\end{minted}
\caption{\texttt{cell} predicates for testing multicolumn properties.}
\label{list:widetests}
\end{listing}

