\chapter{Friends and buddies}
\label{chap:friends}
For a game touted as bringing people together, Pokémon GO tries hard to elide social elements.
Aside from opt-in partners at local raids, there are no in-game means to discover other Trainers.
It is not generally possible to send messages.
Use of real names in Trainer handles is gently discouraged.
Despite this, a Trainer can have up to 450 other Trainers as ``friends''.
How can this capacity be exploited?

When two Trainers agree to friendship, they are assigned a friendship level of zero.
The friendship can progress through four levels, based on the number of calendar
  days the Trainers interact (\autoref{table:friendlevels}).
Reaching a new level awards XP to both Trainers, and the awards for levels 3 and 4 are very substantial.
Interaction can take the form of battling together, trading, a friendly 3x3 match, or
  opening a Gift sent by the other Trainer (\autoref{sec:gifts}).
Addition of and regular gift exchange with other Trainers, facilitated by online friend code parlors,
  is the fastest way to progress through levels.
In the words of Ferris Bueller, ``it's a little silly and childish, but then again, so is [Pokémon GO].''
Damage is enhanced when battling with friends, and winning together
  makes more Premier balls available (these bonuses do not stack).
Higher friendship levels provide large discounts to trades.
\begin{table}[hb]
\centering
\begin{tabular}{llll}
Bonus & Good & Great & Ultra & Best\\
\Midrule
Days & 1 & 7 & 30 & 60\\
XP  & 3,000 & 10,000 & 50,000 & 100,000\\
XP/friend-day & 3,000 & 1,857 & 2,100 & 1,811\\
Damage bonus & 3\% & 5\% & 7\% & 10\%\\
Ball bonus & 0 & 1 & 2 & 4\\
Trade discount & & & & \\
\end{tabular}
\caption{Friendship levels and resulting bonuses}
\label{table:friendlevels}
\end{table}

\section{Gifts}
\label{sec:gifts}
Up to 100 gifts can be acquired from Pokéstops each calendar day.
Until either the daily limit or the maximum capacity of twenty gifts
 is exceeded, spinnning a Pokéstop always provides a Gift.
Buddy Pokémon can provide up to fifteen gifts per day,
  and the Trainer can open up to twenty.
Together, a Trainer can maintain a daily flow of 135 gifts.
If these are all exchanged with new friends, they represents 405,000 XP generated every day.
This would be rather tedious and impersonal; I don't recommend it.
Instead, if you want to quickly gain levels, regularly add a few dozen Trainers.
Take them through at least the Ultra Friend level.
While this falls short of the maximum XP per day, it's much less work, and facilitates use of Lucky Eggs.
You're going to be swimming in items either way.

Remember, only one Trainer needs open a gift to count that calendar day.
Unless I need the items, I try to open my twenty gifts late in the evening,
  preferring Trainers who didn't interact with me that day.
When browsing your list of friends, Trainers with a blue halo have already interacted with you,
  and opening their gift will not advance the friendship state.

\section{Parties}
\label{sec:parties}
Trainers can play as a party, with party-specific challenges and rewards.
They need not be friends, but will require a code and relative proximity.
Any Trainer can create a party, but at least two Trainers must be involved to engage in party play.
A party can have up to four Trainers, and can last up to three hours.
New trainers can replace departures, but the party is dissolved if all other Trainers leave.
Trainers cannot take over a party or form a splinter faction; departure of the host will
  require remaining Trainers to create a new party, and lose any progress.
While party play is active, the other Trainers (and their Buddy Pokémon, if any)
  will be visible on one another's Maps.
When raiding in a party, a Party Power gauge fills, charged by each fast attack used.
When full, charged attacks deal double damage.

\section{Trading}
\label{sec:trades}
A Pokémon can be traded only once across its lifetime.
\textbf{FIXME}

A Best Friend can sometimes become ``lucky'', guaranteeing that the next trade with
  them will result in Lucky Pokémon.
See \autoref{sec:lucky} for more information.

\section{Buddy Pokémon}
\label{sec:buddies}
\textbf{FIXME}

