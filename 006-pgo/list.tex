% oneside ought not be used for printed, bound material!
\documentclass[ebook,10pt,openany,oneside]{memoir}
\title{the big list}
\author{nick black}
\usepackage{graphics}
\usepackage{hyperref}
\usepackage[export]{adjustbox}
\usepackage[letterpaper, landscape, margin=1cm]{geometry}
\usepackage{fontspec}
\defaultfontfeatures{Scale=MatchUppercase,Ligatures=TeX,Renderer=HarfBuzz}
\usepackage{microtype}
\setmainfont{Gentium Book}
\usepackage{longtable}
\newcommand{\ivlev}[4]{#1/#2/#3·#4}
\newcommand\calign[1]{#1}%{\raisebox{-0.25\height}{#1}}

\begin{document}
\noindent{}\today\\
Dankstats for Pokémon GO\\
nick black (\href{mailto:nickblack@linux.com}{nickblack@linux.com}, r/sosodank, pgo: americanpion)\\
\bigskip

A cycle is the minimum number of fast attacks sufficient to launch the charged attack, plus the charged attack itself.
The configuration is that which maximizes the geometric mean of attack, defense, and MHP, where attack and defense are STA and DEF, augmented by $IV_A$ and $IV_D$ respectively, and scaled by CPM.
$Eff_D$ is defense as defined above, scaled by half of any buff the charged attack has on the user's defense.
Bulk is the geometric mean of $Eff_D$ and MHP.
$Eff_A$ is attack as defined above, scaled by half of any buff the charged attack has on the user's attack.
Prod is the geometric mean of $Eff_A$ and attack power over a cycle,
\textit{e} is the energy left after the first cycle, and is not further incorporated (but probably should be).
\%c is the percentage of the first cycle's power delivered by the charged attack, and is not further incorporated.
Dank is the product of the squares of Prod and Bulk divided by the square of turns in the first cycle; it values speed more than most such stats.
Dank+ scales power by the square root of the attack type's ARA, and defense by the square root of the Pokémon's DRA (see \textit{\href{https://nick-black.com/pgo-quantitative.pdf}{Pokémon GO: A Quantitative Approach}}).
STAB is factored into calculations, when applicable.
\bigskip

\footnotesize
\setlength{\tabcolsep}{2pt}
\input{out/stat1500}
\input{out/stat2500}
\input{out/statmax}
\end{document}
