\chapter{Unbounded team selection}
\label{chap:unbounded}

Building a winning team is all about selecting Pokémon that can win.
Part of this is choosing the strongest possible Pokémon.
This is greatly complicated by the fact that a Pokémon's fitness for
 combat is dependent on its opponent, and complicated further by
 the fact that a team's overall fitness is dependent on the makeup
 of the opposing team.
We generally don't know the opposing team when selecting our own.
How, then, can we account for it?

First off, we can break the Damage equation into those parts dependent
 upon opponents, and those which are independent.
Looking at the Damage equation, Power, $Eff_A$, and STAB do not depend
 upon an attacker's opponent.
Likewise $Eff_D$ and MHP for the defender.
The attacker knows neither $Eff_D$ nor MHP, while the defender doesn't know $Eff_A$.
Type effectiveness always requires knowing the opposing type.
We can thus define an independent core fitness evaluation:

\[ F = Eff_A \times P \times M_{STAB} \times Eff_D \times MHP \]

This is roughly equivalent to

\[ F = Mod_A \times P \times M_{STAB} \times Mod_D \times Mod_S \times CPM^3 \]

Of course, a given Pokémon can generally select from multiple Fast and Charged Attacks,
  knowing up to two of the latter at a given time.
Furthermore, Fast and Charged Attacks are not used in equal ratio (there will
  always be multiple Fast Attacks per Charged Attack), and a Shield can
  nullify a Charged Attack.
How do we define $P$ and $M_{STAB}$?
As only one Fast Attack can be known at a time, it's easy enough to define one
  fitness for each possible Fast Attack, leaving out Charged Attacks for now.
All we need do is normalize the Power of the Fast Attack, using the number of
  Turns it requires.

\[ F = Eff_A \times \frac{P_{Fast}}{T_{Fast}} \times M_{STAB-Fast} \times Eff_D \times Eff_S \]

Integrating Charged Attacks is more complex.
First, we can simply normalize the Charged Attack as we did the Fast Attack,
 determine the number of Fast Attacks necessary to launch it:

\[ N_{Fast} = \lceil\frac{E_{Charged}}{E_{Fast}}\rceil \]

and arrive at a weighted average:

\[ P_{cycle} = \frac{N_{Fast} \times P_{Fast} + P_{Charged}}{N_{Fast} \times T_{Fast} + T_{Charged}} \]

This has several problems. It doesn't account for the possibility of multiple
Charged Attacks. It doesn't account for leftover Energy, which will sometimes
be present whenever $E_{Charged}$ is not a multiple of $E_{Fast}$ (i.e. $N_{Fast}$
might fluctuate from one cycle to the next). It ignores the possibility that
the attacker might not use its Charged Attack immediately (and it is often
unwise to throw Charged Attacks as quickly as possible; see
\autoref{chap:strategy}). Finally, it presumes that the attacker can always get
off a full cycle of Fast Attacks followed by a Charged Attack. In reality, the
attacker might be knocked out long before its Charged Attack becomes relevant,
or the defender might use a Shield. It is tempting to ignore Charged Attacks,
but when they land, they tend to dominate our total Damage.
