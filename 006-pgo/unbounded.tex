\chapter{Unbounded team selection}
\label{chap:unbounded}
Trainer Battles make very little use of random numbers (to the best of my
 knowledge, they only show up when checking for a Charged Attack side
 effect).
Their intrigue is instead rooted in ignorance: ignorance of
 the Pokémon you'll face, the order in which you'll face them,
 their statistics (beyond the static properties of the Species),
 their Attacks, and how their Trainer will employ them.

We've established that Pokémon battle fitness is highly dependent
 on opponent details.
We cannot name a globally superior Pokémon---primarily due to Type Effectiveness,
 there is no individual Pokémon that (assuming competent play by both Trainers)
 can defeat all others in even a 1-on-1 contest.
Yet the game provides a per-Pokémon measure of strength in battle: Combat Power (CP).
Obviously no single statistic can account for all possible opponents, but
 we will see that CP is a very questionable assessment for a PvP
 context (and not great for PvE, either).
A Pokémon's CP is defined as:
\[ CP = \max{10, \frac{Mod_A \times \sqrt{Mod_D} \times \sqrt{Mod_S} \times CPM^2}{10}} \]
CP growth is quadratic with respect to $CPM$, linear with $Mod_A$, and
  sublinear with $Mod_D$ and $Mod_S$.
Remember from our Damage equation that $Mod_A$ is divided by $Mod_D$
 to generate one of the factors.
This suggests that $Mod_D$ is undervalued by the CP equation, which
 would seem to suggest that $Mod_A$ is divided by $\sqrt{Mod_D}$.
We will exploit this discrepancy.
MHP isn't used in the Damage equation, but knocking out a Pokémon
 requires inflicting some average Damage $D$ $n$ times,
 where $n = \lceil\frac{MHP}{D}\rceil$.
It is thus similarly undervalued by CP\@.
The quadratic term for CPM makes sense from the perspective of the Damage
 equation, since CPM is used as a factor when calculating Damage in the
 case of both the attacker and defender.
But since CPM is also used to calculate MHP from $Mod_S$, an argument
 can be made that it ought be raised to the third power.
Since $CPM < 1$ for all Levels, $CPM^3 < CPM^2$, and thus it seems
 overvalued by CP\@.

Why would the game use such a flawed measure of power?
A single stat was clearly desired to cover both modes of Pokémon GO battling.
Since Raids allow substitution of defeated Pokémon, but are subject to a timer,
  the ability to defend and absorb Damage is less important than being able to
  quickly inflict damage.
Perhaps this explains the emphasis on attack, but some of the most important
  factors in the Damage equation are left out of CP entirely!
The Power and timing of the Pokémon's Attacks are not present, despite
  dominating the inflicted Damage and overall flow of the battle.
Type effectiveness and STAB have been likewise ignored, as have the
  properties of Shadow Pokémon.
It ought be clear that we must look beyond CP in our selection of Pokémon.

Change to Damage inflicted by an effective Attack is at least 60\%.
Change to Damage inflicted by an ineffective Attack may be as
 low as 37.5\%.
It is probably worth accepting an ineffective Fast Attack if it
 enables an effective Charged Attack, so long as you actually
 get that Charged Attack off.
Early in the battle, this might trick the opponent into leaving
 off a Shield, allowing a .
