\chapter{Damage}
\label{chap:damage}
Each Attack reduces the opponent's Hit Points by an integer greater than zero.
This integer is called (inflicted) Damage, and calculated as a product of several factors.
The factors are different for Gym/Raid/Max Battles and Trainer Battles,
 but the core $D_C$ of the Damage equation is common to all contests.

\section{Typing multipliers}
We must first determine the multiplier due to Typing:

\[ M = M_{STAB} \times M_{T} \]

$M_{STAB}$ is the STAB bonus, equal to 1.2 when the Attack's Type matches any
  Type of the attacker, and 1 otherwise.
$T$ is the number calculated in \autoref{chap:types} using the Attack Type
 and the defender's Type.
$M_{T}$ is $1.6^{T}$.
For completeness I include a $T$ of -4 in \autoref{table:typemult},
 but remember that this is not currently realizable.

\begin{table}
\begin{center}
\begin{tabular}{c r r r r}
  Effectiveness & $M_{T}$ & Δ\% & $M$ w/ STAB & Δ\% \\
\Midrule
  -4 & 0.1526 & -84.74 & 0.1831 & -81.69 \\
  -3 & 0.2441 & -75.59 & 0.2930 & -70.7 \\
  -2 & 0.3906 & -60.94 & 0.4688 & -53.12 \\
  -1 & 0.625 & -37.5 & 0.75 & -25 \\
  0 & 1 & 0 & 1.2 & 20 \\
  1 & 1.6 & 60 & 1.92 & 92 \\
  2 & 2.56 & 156 & 3.072 & 207.2 \\
\end{tabular}
\caption{Typing multipliers with and without STAB}
\label{table:typemult}
\end{center}
\end{table}

Leaving out the hideous $TE = -4$ case, we see that $M$ ranges
 from 0.2441 (a 75.59\% reduction in Damage) to 3.072
 (a 207.2\% increase).
Considered another way, when $TE = -3$ and there is no STAB bonus,
 it takes about four times as many Attacks (and thus time) to subdue
 an opponent than would be expected without typing effects.
When $TE = 2$ and STAB is in play, it takes about one-third as many Attacks.

\section{The Damage equation}

\[ D_C = P \times \frac{Eff_A}{Eff_D} \times M \]

That is, the Power of the employed attack is scaled by the ratio of
 $Eff_A$ to $Eff_D$ and the product of Type Effectiveness multipliers.
Remembering their definitions, this can be rewritten as:

\[ D_C = P \times \frac{Mod_A \times CPM_A}{Mod_D \times CPM_D} \times M \]

which can of course be rewritten as:

\[ D_C = P \times \frac{Mod_A}{Mod_D} \times \frac{CPM_A}{CPM_D} \times M \]

From there, it's just some multipliers, a floor function (to ensure
 an integer), and a min function (to ensure at least 1 point of Damage
 is always inflicted):

\[ D = \min{1, \left\lfloor P \times \frac{Eff_A}{Eff_D} \times M \times M_M \right\rfloor } \]

Since this is a product, a proportional increase in any factor leads to
 the same change.
If we have a Power of 10, a $Mod_A$ of 140, and a CPM of 0.4628 (corresponding
 to Level 12), no STAB bonus, and no Type Effectiveness, $D_C$ is 647.92.
Increasing Power to 12 (a 20\% increase) yields 777.504.
Increasing $Mod_A$ to 168 (a 20\% increase) yields 777.504.
Switching to an Attack matching the attacker's type adds the
 STAB bonus, a 1.2x multiplier, yielding 777.504.
Increasing CPM 20\% to 0.55536 (somewhere between Levels 17 and 17.5; this
 exact change is not possible) yields 777.504.
A 20\% increase of 647.92 is, of course, 777.504.
Note that this is all independent of the opponent.

\section{Opponent-independent Damage}
We can break the Damage equation into those parts dependent
 upon opponents, and those which are independent.
Looking at the Damage equation, Power, $Eff_A$, and STAB do not depend
 upon an attacker's opponent.
Likewise $Eff_D$ and MHP for the defender.
The attacker knows neither $Eff_D$ nor MHP, while the defender doesn't know $Eff_A$.
Type effectiveness always requires knowing the opposing type.
We can thus define an independent core fitness evaluation:

\[ F = Eff_A \times P \times M_{STAB} \times Eff_D \times MHP \]

This is roughly equivalent ($MHP \approx Mod_S * CPM$) to

\[ Mod_A \times P \times M_{STAB} \times Mod_D \times Mod_S \times CPM^3 \]

Of course, a given Pokémon can generally select from multiple Fast and Charged Attacks,
  knowing up to two of the latter at a given time.
Furthermore, Fast and Charged Attacks are not used in equal ratio (there will
  always be multiple Fast Attacks per Charged Attack), and a Shield can
  nullify a Charged Attack.
How do we define $P$ and $M_{STAB}$?
As only one Fast Attack can be known at a time, it's easy enough to define one
  fitness for each possible Fast Attack, leaving out Charged Attacks for now.
All we need do is normalize the Power of the Fast Attack, using the number of
  Turns it requires.

\[ F = Eff_A \times \frac{P_{Fast}}{T_{Fast}} \times M_{STAB-Fast} \times Eff_D \times Eff_S \]

Integrating Charged Attacks is more complex.
First, we can simply normalize the Charged Attack as we did the Fast Attack,
 determine the number of Fast Attacks necessary to launch it:

\[ N_{Fast} = \lceil\frac{E_{Charged}}{E_{Fast}}\rceil \]

yielding the total Power of a cycle, known by the community as Total Damage Output (TDO):

\[ P_{cycle} = N_{Fast} \times P_{Fast} + P_{Charged} \]

We normalize this:

\[ F_{cycle} = \frac{P_{Cycle}}{N_{Fast} \times T_{Fast} + T_{Charged}} \]

There are two turns per second. Multiplying $F_{cycle}$ by two yields a
  stat the community calls Damage Per Second (DPS).

This has several problems. It doesn't account for the possibility of multiple
Charged Attacks. It doesn't account for leftover Energy, which will sometimes
be present whenever $E_{Charged}$ is not a multiple of $E_{Fast}$ (i.e. $N_{Fast}$
might fluctuate from one cycle to the next). It ignores the possibility that
the attacker might not use its Charged Attack immediately (and it is often
unwise to throw Charged Attacks as quickly as possible; see
\autoref{chap:strategy}). Finally, it presumes that the attacker can always get
off a full cycle of Fast Attacks followed by a Charged Attack. In reality, the
attacker might be knocked out or substituted long before its Charged Attack
becomes relevant, or the defender might use a Shield. It is tempting to ignore
Charged Attacks, but when they land, they tend to dominate our total Damage.

If we expand over multiple cycles of Fast and Charged Attack, we can
 generalize to situations with excess Energy. We know $E_{Charged} \times
 E_{Fast}$ will be a multiple of both the generated and consumed Energy, so
 simply consider $E_{Charged}$ cycles, each of which throws an average of
 $\frac{E_{Charged}}{E_{Fast}}$ Fast Attacks followed by one Charged Attack.

\[ P_{cycles} = E_{Charged} \times (\frac{E_{Charged}}{E_{Fast}} \times P_{Fast} + P_{Charged}) \]

 The number of Fast Attacks in a cycle is actually always either
 $\lfloor\frac{E_{Charged}}{E_{Fast}}\rfloor$
 or $\lceil\frac{E_{Charged}}{E_{Fast}}\rceil$ (iff $E_{Charged}$ is a multiple of
 $E_{Fast}$, these two expressions are equal, and the number of Fast Attacks
 per Charged Attack is constant). Normalize for the total time:

\[ F_{cycles} = \frac{P_{cycles}}{E_{Charged} \times (\frac{E_{Charged}}{E_{Fast}} \times T_{Fast} + T_{Charged})} \]

This has only exacerbated one of the problems we mentioned before: we may
  not get to throw all these Attacks!
Suppose we have some constant chance $0 < L_{KO} \leq 1$ of being knocked out or
 substituted following each Attack we launch, the probability of being
 in to throw the $N$th Attack is $(1 - L_{KO})^{N-1}$.
We could thus define an expected damage from Attack $N$:

\[ E_D(N) = \overline{P} \times (1 - L_{KO})^{N-1} \]

and a cumulative expected damage through $N$ Attacks:

\[ E_{TD}(N) = \sum^N_{i=1} \overline{P} \times (1 - L_{KO})^{i-1} \]

This is a geometric series where $a = \overline{P}$ and $r = 1 - L_{KO}$.
Since $0 \leq r < 1$, this series converges to

\[ E_{TD}(\infty) = \sum^\infty_{i=1} \overline{P} \times r^{i-1} = \frac{\overline{P}}{1 - L_{KO}} \]

Of course, our chance of being knocked out is usually not constant across
Attacks, but rather an immediate function of our remaining HP and any
Damage we are about to absorb.
