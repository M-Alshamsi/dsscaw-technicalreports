\chapter{CP-bounded team selection}
\label{chap:bounded}
Several leagues (and some Research tasks) place a ceiling on the CP of
 participating Pokémon.
We have established that CP is a poor proxy for PvP performance, and thus
 expect that we might find strong candidates based on where CP fails.
If CP does not accurately represent a given Pokémon's chance to win, we ought
 be able to exploit this by assembling teams using undervalued Pokémon.
We first take the sorted list of Pokémon developed in \autoref{chap:unbounded},
 and remove all entries whose CP exceed the CP ceiling.
We then select a team using the strategies of that chapter. Since
 that list was sorted according to a general strength function, it
 ought apply in this reduced context.

\section{Optimal IVs}
For a given CP bound and Form, there is some optimal set of Pokémon Levels
 and Individual Vectors.
Higher IV components might require a lower level than lower components to
 come in under the same ceiling.
It is not possible to provide a global rule; optimal IVs are strongly
 dependent upon the base stats, and weakly dependent upon the opponent.
What can be said is that IVs add less value to a high base stat than a low
 base stat, and that higher levels provide diminishing returns.
For a base stat of 100, a corresponding IV component of 15 represents a
 15\% improvement to $Eff_A$.
For a base stat of 300, the same IV represents only a 5\% improvement.
A Level 10 Pokémon advancing to level 10.5 enjoys a 2.47\% improvement to
 $Eff_A$ (and also $Eff_D$ and $Eff_S$), paying a price of 5\% increase
 in CP.
Advancing from 30 to 30.5 represents only a 0.42\% improvement (and a
 0.83\% increase in CP).

The maximum level for which a 15--15--15 IV comes in under the bound is the
 minimum possible optimal level, since any lesser level would be strictly less
 powerful.
The maximum level for which a 0--0--0 IV comes in under the bound is the
 maximum possible optimal level, as it is not possible to gain another level
 while remaining within the bound.
Any IV with all three components less than or equal to another IV, but bounded
 by the same level, is strictly inferior to the second IV, and can be discarded.
Compute $Eff_A$, $Eff_D$, and $MHP$ using the IVs and corresponding levels.
Any resulting set with all three components less than or equal to some other set's
 is strictly inferior to the second, and can be discarded.
This captures $MHP$'s floor function.
What remains is an optimal frontier of at least one solution,
 where arguments can be made for each depending on how one values attack,
 defense, and stamina, especially in the context of different opponents.

Before taking Power and Type into the question, let's examine the optimal
 solutions for a genus using the 1500 and 2500 CP bounds, using the
 geometric mean of $Eff_A$, $Eff_D$, and $MHP$ as our global fit function.
Timburr (134 ATK, 87 DEF, and 181 STA) evolves into
  Gurdurr (180 ATK, 134 DEF, and 198 STA), which evolves into
  Conkeldurr (243 ATK, 158 DEF, 233 STA).
Their base products are 2,110,098 (Timburr), 4,775,760 (Gurdurr),
  and 8,945,802 (Conkeldurr).
The geometric means are 128.263, 168.402, and 207.590.
Poor Timburr isn't really suited for the common PvP Leagues,
  and does best with a 15--15--15 at Level 50 (CP 1487).
Gurdurr can hold its own under a 1500 CP ceiling, and its
  optimal IVs are 1--15--15 at Level 26 (CP 1496).
It falls short of the Master League's 2500, though, where
  15--15--15 and 15--15--14 are functionally equivalent
  at Level 50 (both result in MHP of 178) despite
  respective CPs of 2452 and 2447.
It is impossible to distinguish between a Level 16.5 Conkeldurr at
  3--15--15 or 3--15--14 (134 MHP either way), though
  the CPs are 1500 and 1497.
In the Master League, Conkeldurr wants Level 28 0--15--12 with
  a CP of 2499.
Take two things away from this:
\begin{itemize}
\item The optimal IVs, and even the number of equivalent optimal IVs, can change across evolution.
  This is largely independent of any global fitness function.
\item Hitting the ceiling does not guarantee optimal IVs for most fitness functions.
  More generally, under most fitness functions, higher CP does not necessarily imply superior IVs.
    Gurdurr has numerous solutions yielding 1500 CP (Level 28 3--2--0, for instance, with geometric
    mean 120.015), but Level 26 1--15--15 beats it at 1496 (121.968).
\end{itemize}
There are many fitness functions. In the Gurdurr example given previously, 3--2--0 loses
    to 1--15--15 at geometric mean, but has an $Eff_A$ of 129.36, comfortably above the latter's 123.29.
 Does this compensate for six fewer hit points (145 vs 139) and lower $Eff_D$ (101.49 vs 96.136)?
 Let's pit them against one another, using Low Kick (Power of 4, neutral type effectiveness, STAB bonus).
 Our Level 28 competitor strikes for six damage with each attack:
    \[ \lfloor \frac{129.36 \times 4 \times 1.2}{101.49} \rfloor = \lfloor 6.118 \rfloor = 6 \]
 The supposedly optimal Gurdurr strikes for six as well:
    \[ \lfloor \frac{123.29 \times 4 \times 1.2}{96.136} \rfloor = \lfloor 6.156 \rfloor = 6 \]
 At 145 HP, Gurdurr goes down on the 25th attack.
 Its opponent's 139 HP absorbs 23 attacks before falling to the 24th, but this could
    have easily gone the other way.
 If the attack's Power were 10, both hit for nine damage, and both fall after nine attacks.

\section{Team assembly}
Does this extend to teams?
At first, it seems that it might: after all, a team's overall fitness
 depends on the entire opposing team, involving still more unknowns.
We will see that the team mechanism actually allows us to mitigate certain
 unknowns, while adding new ones.
Comparing teams is highly nontrivial, and this greater complexity admits
 further opportunities for strategy and cunning.

