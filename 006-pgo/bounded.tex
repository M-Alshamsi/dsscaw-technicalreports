\chapter{CP-bounded team selection}
\label{chap:bounded}
Several leagues (and some Research tasks) place a ceiling on the CP of
 participating Pokémon.
We have established that CP is a poor proxy for PvP performance, and thus
 expect that we might find strong candidates based on where CP fails.
If CP does not accurately represent a given Pokémon's chance to win, we ought
 be able to exploit this by assembling teams using undervalued Pokémon.
We first take the sorted list of Pokémon developed in \autoref{chap:unbounded},
 and remove all entries whose CP exceed the CP ceiling.
We then select a team using the strategies of that chapter. Since
 that list was sorted according to a general strength function, it
 ought apply in this reduced context.

\section{Optimal IVs}
For a given CP bound and Form, there is some optimal set of Pokémon Levels
 and Individual Vectors.
Higher IV components might require a lower level than lower components to
 come in under the same ceiling.
It is not possible to provide a global rule; optimal IVs are strongly
 dependent upon the base stats, and weakly dependent upon the opponent.
What can be said is that IVs add less value to a high base stat than a low
 base stat, and that higher levels provide diminishing returns.
For a base stat of 100, a corresponding IV component of 15 represents a
 15\% improvement to $Eff_A$.
For a base stat of 300, the same IV represents only a 5\% improvement.
A Level 10 Pokémon advancing to level 10.5 enjoys a 2.47\% improvement to
 $Eff_A$ (and also $Eff_D$ and $Eff_S$), paying a price of 5\% increase
 in CP.
Advancing from 30 to 30.5 represents only a 0.42\% improvement (and a
 0.83\% increase in CP).

The maximum level for which a 15--15--15 IV comes in under the bound is the
 minimum possible optimal level, since any lesser level would be strictly less
 powerful.
The maximum level for which a 0--0--0 IV comes in under the bound is the
 maximum possible optimal level, as it is not possible to gain another level
 while remaining within the bound.
Any IV with all three components less than or equal to another IV, but bounded
 by the same level, is strictly inferior to the second IV, and can be discarded.
Compute $Eff_A$, $Eff_D$, and $MHP$ using the IVs and corresponding levels.
Any resulting set with all three components less than or equal to some other set's
 is strictly inferior to the second, and can be discarded.
This captures $MHP$'s floor function.
What remains is an optimal frontier of at least one solution,
 where arguments can be made for each depending on how one values attack,
 defense, and stamina, especially in the context of different opponents.

\section{Team assembly}
Does this extend to teams?
At first, it seems that it might: after all, a team's overall fitness
 depends on the entire opposing team, involving still more unknowns.
We will see that the team mechanism actually allows us to mitigate certain
 unknowns, while adding new ones.
Comparing teams is highly nontrivial, and this greater complexity admits
 further opportunities for strategy and cunning.

