\chapter{Tactics\label{chap:strategy}}
This chapter addresses tactics for 3x3 (the only real tactics for Nx1 are (1)
  press the charged attack button when full and (2) try not to get hit).
The three subsequent chapters cover team selection, but the topics are intertwined.
A Trainer's Pokémon enable tactics, and tactics guide Pokémon selection.

One of the greatest advantages one can gain is knowledge.
Know Pokémon well, especially those on your team and those you regularly see.
Know what moves they can learn, and how hard they throw them (Power and $Eff_A$).
Knowing their bulk ($Eff_D$ times MHP) can help decide which charged attack to use.
Know type relations cold, and be able to calculate type effectiveness on the fly.
The flipside of this is exploitation of ignorance.
Opposing Trainers are less likely to know Pokémon outside the ``meta'',
 especially those with uncommon typings.
A Pokémon can only know two charged attacks at a time, but if
 chosen from a move pool of three or more charged, even an informed opponent
 can't know what's coming until it's thrown.

\begin{tipbox}[title=A tip regarding battle UI]
Without sufficient energy, pressing the charged attack control employs a fast attack.
If you know you want to throw a charged attack as soon as possible, press its control until thrown.
\end{tipbox}

Take two charged attacks having equal power-per-energy and type, but different power.
The attack delivering more power (and requiring more energy) is beneficial
 when it can achieve a knockout in one hit, but the attack with less power cannot.
This can happen when a new opponent is subbed in, or when the Pokémon is brought back
 after having been subbed out with significant charge.
It also benefits from only one application of the floor function.
Otherwise, the cheaper attack deals comparable damage over time, but deals a portion
 of it earlier, while being less vulnerable to shielding.

Having two charged attacks is a tremendous advantage over only one.
If they are different types, you can gain type advantage over a wide range of opponents.
Since you choose which one to throw on the fly, resistances are much less of a concern.
\textbf{FIXME}

Change to Damage inflicted by an effective Attack is at least 60\%.
Change to Damage inflicted by an ineffective Attack may be as
 low as 37.5\%.
It is probably worth accepting an ineffective Fast Attack if it
 enables an effective Charged Attack, so long as you actually
 get that Charged Attack off.
Early in the battle, this might trick the opponent into leaving
 off a Shield, allowing a \textbf{FIXME}.

\subsection{Catching charged attacks}
When a Pokémon drops to very low health (one or two more fast attacks will knock
 them out), they can be subbed out.
Later, when you know a charged attack to be imminent, bring them back in.
If the charged attack is spent knocking out the original Pokémon, its total
 damage is greatly lessened, and you can immediately return to the Pokémon
 you had up.
If you can get an effective fast attack out of the sacrifice, all the better.
This requires sufficient time between the sub out and the catch, and that the
 opponent doesn't knock out your sacrifice Pokémon with fast attacks.

\subsection{Saving charged attacks}
If your Pokémon has almost enough energy to throw a charged attack, but is close to fainting, it can be useful to sub them out.
When they come back, the situation might have changed so that they can get the attack off,
 rather than wasting the builtup energy.
A downside is that an incoming charged attack will probably inflict more total
 damage on a healthy Pokémon than one with little HP to lose.

\subsection{Charged move subterfuge}
It's best to keep an opponent guessing as to your charged moves.
With two moves, an opponent rarely knows which is coming with certainty.
Even with a single move, there is some room for deception.
By holding onto a charged attack longer than necessary, a Pokémon can build
  up energy towards a subsequent attack, and hint towards an attack
  that requires more energy (and thus presumably packs more power).
An opponent that would have absorbed the relatively weak charged attack
  might be induced to blow a shield instead.
This can be particularly effective when the opponent can be drawn down to low HP
  and has one shield left.
The charged attack comes, and is mistaken for a very powerful attack.
A shield is deployed, with the reasoning that the attacked Pokémon can
  do substantial damage with their remaining health, as there isn't
  enough time for a second attack to be charged.
Instead, a second attack comes almost immediately; the Pokémon goes
  down by necessity, and no shields remain.

\subsection{Substitutions\label{subsec:substitutions}}
When the active Pokémon is knocked out, the Trainer selects an unfainted
  Pokémon to replace it (if there is only one such Pokémon, it comes in
  automatically; if there are none, the match is over).
Before a knockout, the Trainer can call for a substitution using any
  unfainted Pokémon.
This starts a substitution cooldown timer of fifty seconds; the Trainer
  cannot perform an early substitution during this time.
Under what circumstances ought early substitutions be made?
Remember that subbing out a Pokémon removes all buffing (\autoref{sec:buffs})
  active for that Pokémon (though not buffs it effected).

``Super Effective!'' over one's Pokémon is a frightening message.
It is natural to want to pull them out, ideally bringing in a teammate resistant
  to the revealed opponent's attacks and effective against their typing.
If the opponent can freely substitute, and has a counter to the Pokémon brought in,
  our Trainer is in a world of shit.
Their substituted Pokémon is going to get cooked, and it is unlikely to inflict much damage on the countering Pokémon.
Furthermore, the original threat is still lurking, our original Pokémon is still at a disadvantage,
 and we cannot sub further.
There's no faster way to lose a match.
How, then, to deal with a highly disadvantageous initial matchup?

One can always accept the loss of their lead Pokémon, trying to take some shields and HP with them.
This is not unreasonable if the Trainer has a strong counter at position two or three
  that can come out and mow down the initial Pokémon.
The opponent will of course be able to bring in their own counter when their lead goes down.
The opponent might substitute immediately, protecting their lead (and any charged energy); the Trainer will be able to substitute in turn, possibly locking in a mismatch.
Two Trainers enter a match knowing nothing of the other team's composition.
Losing a Pokémon has the silver lining of freely choosing among your remaining team,
  while knowing at least one member of the opposition.
Substitution has the same benefit.
A chained team, where each Pokémon hard counters the counter of another, can accept such sacrifices.

If the Pokémon brought in is particularly resilient to damage, it might be unlikely that
 the opponent has a good counter available.
By that logic, why not use such a Pokémon as lead, and make the substitution unnecessary?
The lead goes in completely uncertain of their opponent (though the Trainer might have suspicions).
Their charged attacks can always be shielded.
For this reason, I prefer a lead with minimal weaknesses (even if it means few resistances),
 strong fast attacks (perhaps at the expense of energy) with few resistances (even if it means
 few strengths), and low cost charged attacks (despite low power). 
Normal and Ghost are obviously good candidates here.
Water has some of the best attacks, and several attractive strengths, but Electric and
 especially Grass counters can wipe out Water.
Grass pairs well with Water, the two eliminating most of the other's weaknesses;
 don't sleep on Ludicolo.
Take a look at \autoref{table:defenders}, and familiarize yourself with the first
 and second pages.
I like a lot of hit points in my lead, as they won't be dealing extreme damage.

The intended third Pokémon is coming into a situation with minimal shields available
 to either Trainer.
There's nothing they can do about a mismatch.
They're almost certain to get hit hard, and likely to get their charged attacks in.
Here I don't care about weaknesses; the opponent has limited substitution potential,
 and I don't expect to be around very long anyway.
I want lots of attack---Shadow Pokémon are useful here---and fast attacks that quickly enable my powerful charged attacks.
This Pokémon absolutely must have two strong attacks of different type; it's essential that the charged attack not be resisted.
It's best if they both have STAB\@.
The overpower of charged attacks relative to stamina makes trying to win this via defense a bad idea.

Substituting out a Pokémon with sufficient energy for a charged move (or even two) is a powerful technique.
By doing so, a Trainer has more control over when to use the accumulated energy.
Why waste a knockout blow on an opponent that's already on the ropes?
This can be particularly useful when a fast move is very effective against the current opponent,
 but your charged move is not.
Save that energy for a more profitable matchup.
Of course, the Pokémon you would have knocked out might get a charged attack off on your sub,
 and/or a hard counter might be brought in:
 this is a gamble until you know the opponent's full lineup.
On this note, when deciding whether to shield, be sure to account for built up energy on the opponent's bench.

\subsection{Charged move timing\label{subsec:cmt}}
\begin{tipbox}[title=Warning! Achtung! \begin{chinese}危险!\end{chinese} ¡Peligro! \begin{japanese}危険!\end{japanese} \textit{Tulaga faigata!}]
This subsection contradicts longstanding claims of the Pokémon GO community, repeated by
  people far more experienced than me, whom I will probably never defeat in PvP.
Nonetheless, I'm pretty sure they're all wrong, and I'm right.
You'll have to decide for yourself.
Then again, that's always the case.
\end{tipbox}
Recall from \autoref{chap:battle} the mechanics of charged moves: if Pokémon A has a fast move ongoing,
 and Pokémon B throws a charged attack, the charged attack brings the fast attack to early completion.
Assuming Pokémon A isn't knocked out by the charged attack, the damage due their fast attack is
 immediately inflicted, and the energy immediately built up.
On the next turn, both Pokémon can throw a new attack.
This much is accepted by everyone.

Over and over, one will hear that ``charged moves ought be thrown on the natural last turn of an
 opponent's fast move, or the opponent gains free turns.''
Examples include EmpoleonDynamite's ``\href{https://pvpoke.com/articles/strategy/guide-to-fast-move-registration/}{Guide to Fast Move Mechanics}''
 on \href{https://pvpoke.com}{pvpoke.com} (2022), FlarkeFiasco's r/TheSilphArena post ``\href{reddit.com/r/TheSilphArena/comments/fvu62a/charge\_move\_timing\_yes\_it\_really\_does\_matter/}{Charge Move Timing---Yes, it *really* does matter}'' (2020),
 avrip's ``\href{https://pokemongohub.net/post/pvp/mastering-the-turn-system-in-pokemon-go-pvp/}{Mastering the Turn System in Pokémon GO PVP}'' on GOHub (2024),
 and Wallower's YouTube video ``\href{https://www.youtube.com/watch?v=pAtCo8xg700}{GBL Advanced Mechanics 1: Charged Move Timing}'' (2021).
Fine people, all, and far more accomplished than I will ever be.
But this claim is (mostly) false, and has been cargo culted for years.
Following the rule has no costs if done correctly, but it brings no advantage, either.

Suppose that Pokémon B is running Incinerate, a five-turn fast move with twenty power and twenty energy.
At the start of turn $T_1$, Pokémon B has ten energy and 70 HP\@.
Pokémon A has fifty energy and 100 HP\@.
Pokémon B throws Incinerate.
It will normally conclude on $T_5$, leaving Pokémon B with thirty energy, and Pokémon A with 80 HP\@.
On $T_2$, Pokémon A throws a charged attack (violating the rule) costing 45 energy, hitting Pokémon B for 40 HP\@.
Incinerate is brought to an early conclusion.
At the end of $T_2$, Pokémon B has 30 HP and 30 energy, Pokémon A has 5 energy and 80 HP, and both are ready to throw a new attack.

The rule mandates that Pokémon A not throw the charged attack until $T_5$.
In that case, the fast attack concludes on $T_5$, uninterrupted by the charged attack.
The charged attack's damage still takes precedence---not that it matters in this case---and at the
  end of $T_5$, Pokémon A has 5 energy and 80 HP, while Pokémon B has 30 HP and 30 energy.
Both are ready to select a new attack at $T_6$.
The two situations are \textit{exactly the same} except that five turns have expired instead of two.
Both Pokémon end up with the same energy as they had above, the same HP, the same number of shields,
  the same number of Pokémon remaining, and both are free to launch a new attack.
Aside from the absolute clock and possibly the substitution timer, an observer cannot distinguish
  between the two resulting states.
Pokémon B has not emerged with more energy, has not thrown any more attacks, and is not in a better
  position to throw its next attack.
Where is this ``free attack'' it was to gain?
There is no such thing.

I said the claim was ``mostly false'', because there is a relevant case:
  if Pokémon A has a fast attack that fits within the turns required by Pokémon B's attack,
  it absolutely ought be employed before the charged attack.
Let's assume Pokémon A employs Poison Jab, a two-turn fast attack.
When Pokémon B throws Incinerate at $T_1$, Pokémon A can throw Poison Jab.
It concludes on $T_2$, dealing damage and charging energy, without interrupting Incinerate.
On $T_3$, Pokémon A can launch Poison Jab a second time, concluding on $T_4$.
Finally, at $T_5$, Pokémon A throws its charged attack.
The result when $T_6$ rolls around is that both stand ready to throw new attacks, like before,
  but Pokémon A has the benefit of two extra fast attacks.
Compared to throwing the charged attack at $T_2$, or waiting until $T_5$,
  Pokémon A has inflicted more damage and built up more energy.

Now, if Pokémon A has a three-turn attack like Astonish, its options are more limited.
It can (and should) throw Astonish at $T_1$, concluding at $T_3$.
Throwing Astonish a second time won't see it conclude until $T_6$.
Assume Pokémon B throws a second Incinerate at $T_6$.
If Pokémon A throws its charged attack at $T_7$, Pokémon B has indeed landed two Incinerates going into $T_8$,
  doubling inflicted damage and energy charged.
Of course, Pokémon A has landed two Astonish attacks in addition to its charged attack, so the
  second Incinerate wasn't quite ``free''.
