\chapter{Tactics}
\label{chap:strategy}
One of the greatest advantages one can gain is knowledge.
Know Pokémon well, especially those on your team and those you regularly see.
Know what moves they can learn, and how hard they throw them (Power and $Eff_A$).
Knowing their bulk ($Eff_D$ times MHP) can help decide which charged attack to use.
Know type relations cold, and be able to calculate type effectiveness on the fly.
The flipside of this is exploitation of ignorance.
Opposing Trainers are less likely to know Pokémon outside the ``meta'',
 especially those with uncommon typings.
A Pokémon can only know two charged attacks at a time, but if
 chosen from a move pool of three or more charged, even an informed opponent
 can't know what's coming until it's thrown.

\begin{tipbox}[title=A tip regarding battle UI]
Without sufficient energy, pressing the charged attack control employs a fast attack.
If you know you want to throw a charged attack as soon as possible, press its control until thrown.
\end{tipbox}

Take two charged attacks having equal power-per-energy and type, but different power.
The attack delivering more power (and requiring more energy) is beneficial
 when it can achieve a knockout in one hit, whereas the attack with less power cannot.
This can happen when a new opponent is subbed in, or when the Pokémon is brought back
 after having been subbed out with significant charge.
It also benefits from only one application of the floor function.
Otherwise, the cheaper attack deals comparable damage over time, but deals a portion
 of it earlier, while being less vulnerable to shielding.

Having two charged attacks is a tremendous advantage over only one.
If they are different types, you can gain type advantage over a wide range of opponents.
Since you choose which one to throw on the fly, resistances are much less of a concern.
\textbf{FIXME}

Change to Damage inflicted by an effective Attack is at least 60\%.
Change to Damage inflicted by an ineffective Attack may be as
 low as 37.5\%.
It is probably worth accepting an ineffective Fast Attack if it
 enables an effective Charged Attack, so long as you actually
 get that Charged Attack off.
Early in the battle, this might trick the opponent into leaving
 off a Shield, allowing a \textbf{FIXME}.

\subsection{Catching charged attacks}
When a Pokémon drops to very low health (one or two more fast attacks will knock
 them out), they can be subbed out.
Later, when you know a charged attack to be imminent, bring them back in.
If the charged attack is spent knocking out the original Pokémon, its total
 damage is greatly lessened, and you can immediately return to the Pokémon
 you had up.
If you can get an effective fast attack out of the sacrifice, all the better.
This requires sufficient time between the sub out and the catch, and that the
 opponent doesn't knock out your sacrifice Pokémon with fast attacks.

\subsection{Saving charged attacks}
If your Pokémon has almost enough energy to throw a charged attack, but is close to fainting, it can be useful to sub them out.
When they come back, the situation might have changed so that they can get the attack off,
 rather than wasting the builtup energy.
A downside is that an incoming charged attack will probably inflict more total
 damage on a healthy Pokémon than one with little HP to lose.

\textbf{Swap leadership and safe swaps}

\textbf{shield baiting}

\textbf{optimal charged move timing}

\section{Substitutions}
\label{sec:substitutions}
When the active Pokémon is knocked out, the Trainer selects which unfainted
  Pokémon will replace it (if there is only one such Pokémon, it comes in
  automatically; if there are none, the match is over).
Before a knockout, the Trainer can call for a substitution using any
  unfainted Pokémon.
This starts a substitution cooldown timer of fifty seconds; the Trainer
  cannot perform an early substitution during this time.
Under what circumstances ought early substitutions be made?

``Super Effective!'' over one's Pokémon is a frightening message.
It is natural to want to pull them out, ideally bringing in a teammate resistant
  to the revealed opponent's attacks and effective against their typing.
This will initiate the cooldown period.
If the opponent can freely substitute, and has a counter to the Pokémon brought in,
  the first Trainer is in a world of shit.
Their substituted Pokémon is going to get cooked, and it
 is unlikely to inflict much damage on the countering Pokémon.
Furthermore, the original threat is still lurking, and our original Pokémon is still at a disadvantage.
There's no faster way to lose a match.

\textbf{FIXME}
