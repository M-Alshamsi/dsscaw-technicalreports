\chapter{Meging and Maxing}
\label{chap:megmax}

\section{Mega forms}
\label{sec:mega}
Species with corresponding Mega Energy can take on a Mega form for eight hours.
Only one of a Trainer's Pokémon can be in its Mega form at any given time---Mega
  evolving Pokémon will cause any existing Mega form to revert.
Upon reversion, the Pokémon has a rest period before it can Mega evolve again.
The Mega evolution revives and fully heals the evolved Pokémon, and persists across
  fainting (i.e. the evolved Pokémon can be revived into the Mega form).
Fainting does not stop the eight hour counter.
Mega Pokémon cannot be used in Max Battles, and cannot be traded while in Mega form.
Mega levels are reset upon a trade.
\textbf{FIXME: bonuses to raids, candy}
\input{out/mega}

\section{Primal forms}
\label{sec:primal}
Primal reversion is very similar to Mega evolution.
\textbf{FIXME: bonuses, differences}
\input{out/primal}

\section{Crowned forms}
\label{sec:crowned}
Crowning is a reversible, persistent evolution available to Zacian and Zamazenta
 so long as they know the charged attack Iron Head.
Upon coronation, stats are modified, and Iron Head is replaced with Behemoth Blade (Zacian)
  or Behemoth Bash (Zamazenta).
This new attack cannot be changed with even an Elite Charged TM\@.
A Crowned form can fight in Max Battles, with Behemoth Blade as its Max Attack.
This Max Attack follows the same rules for leveling and power as any other.
Crowning consumes 1,000 Crown Energy, but once a Pokémon has been crowned,
  that Pokémon can freely change between the base and crowned
  forms\footnote{I'm not sure why one would uncrown. Perhaps to change charged attacks?}.
\input{out/crowned}

\section{Dynamax forms}
\label{sec:dmax}
Dynamax forms have access to three Max Moves: Max Attack, Max Guard, and Max Spirit.
Max Moves have four levels: Locked, 1, 2, and 3.
A freshly acquired (via capture or trade) Dynamax Pokémon has Max Attack at level 1,
  while Max Guard and Max Spirit are both locked.
The Max Attack known depends on the Pokémon's fast attack.
Max Move levels persist across evolution and use of Fast TMs.
\textbf{FIXME: how/when Max Moves are used}
\input{out/dynas}

\section{Gigantamax forms}
\label{sec:gmax}
Gigantamax forms are similar to Dynamax forms to the degree that they can
  employ Max Moves in Max Battles.
Unlike Dynamax forms, each Gigantamax knows a species-specific attack
  called a G-Max Move, which does not change to match the fast attack.
G-Max attacks of level 1, 2, and 3 have power of 350, 400, and 450
  respectively, 100 more than Dynamax attacks of the same level.
Gigantamax forms have their own visual representation.
\input{out/gigantas}

\section{Fused forms}
\label{sec:fusion}
Certain Pokémon can be reversibly fused, yielding a new form.
The fusion process requires 1,000 units of Fusion Energy particular to the output\footnote{They all kinda sound like Mountain Dew flavors.},
 and 30 Candy of each input species (\autoref{table:fusion}).
There is no cost to undo the fusion.
Any powering up that is performed on the fused form will apply only to the primary
  Pokémon if the fusion is undone.
Fusion replaces the first charged attack of the primary Pokémon.
\begin{table}[ht]
\centering
\begin{tabular}{llll}
  Primary & Secondary & Catalyst & Output & Attack\\
\Midrule
  Kyurem & Zekrom & Volt Fusion & Black Kyurem & Freeze Shock \\
  Kyurem & Reshiram & Blaze Fusion & White Kyurem & Ice Burn\\
  Necrozma & Solgaleo & Solar Fusion & Dusk Mane Necrozma & Sunsteel Strike\\
  Necrozma & Lunala & Lunar Fusion & Dawn Wings Necrozma & Moongeist Beam\\
\end{tabular}
\end{table}
\input{out/fused}
