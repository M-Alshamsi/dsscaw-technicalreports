\clearpage
\chapter{Foreword}

\noindent{}Pokémon and Pokémon character names are trademarks of Nintendo.
This book and its author are in no way associated with the Nintendo, Niantic,
  or Scopely corporations.\\
\\
\noindent{}The text is specific to Pokémon GO, and focuses on the free-to-play and PvP
  ``League'' aspects of that game, though many details
  apply to the rest of the Pokémon family.
In particular, it is based on my experimentation with and analysis of
 the 0.363.2 Android release, using decompilation, debuggers, and
 traffic analysis.
Whenever possible, I have relied exclusively on official Niantic communications
 and my own research.
I have noted all data not personally validated.\\
\\
\noindent{}Changes are frequently made to the game, some of them quite fundamental.
This text documents PGO as it is played in July 2025.
I have not generally explored variances with historical gameplay.
Mathematical sophistication is not required.
A basic familiarity with algebra ought suffice.
I resort to calculus only once.
Only \autoref{chap:simul} can be called ``technical''.\\
\\
\noindent{}Much of the information herein will be old news to experienced
 PGO players, but I hope to provide both a rigorous (if opinionated) reference and an ordered collection of wisdom.
The community established a large body of knowledge long before I
 began playing in March of 2025.
Still, some of my theories regarding team selection and battle strategy might
  be new to you.
My novel contributions are primarily found in \autoref{chap:unbounded},
  \autoref{chap:bounded}, and \autoref{chap:simul}.
I will not be recommending Pokémon, nor teams thereof---ephemeral information at best---but
 instead giving you the tools to make your own creative decisions.\\
\\
\noindent{}While writing this book, I extensively consulted the
  GamePress Pokémon GO Wiki\footnote{\url{https://pokemongo.gamepress.gg}},
  the Fandom Pokémon GO Wiki\footnote{\url{https://pokemongo.fandom.com/}},
  the Pokémon GO Database\footnote{\url{https://db.pokemongohub.net/}},
  and the work of the now-disbanded Silph Road research group.
In particular, all bitmaps were sourced from the two wikis.
All graphs, diagrams, and text are my original work.\\
\\
\noindent{}Please send corrections and feedback to 
  \href{mailto:nickblack@linux.com}{nickblack@linux.com}.
