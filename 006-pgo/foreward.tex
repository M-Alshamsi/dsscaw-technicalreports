\clearpage
\chapter{Foreward}

\noindent{}Pokémon and Pokémon character names are trademarks of Nintendo.
This book and its author are in no way associated with the Nintendo, Niantic,
  or Scopely corporations.\\
\\
\noindent{}The text is specific to Pokémon GO, and focuses on the free-to-play and PvP
  ``League'' aspects of that game, though many details
  apply to the rest of the Pokémon family.
In particular, it is based on my experimentation with and analysis of
 the 0.363.2 Android release, using decompilation, debuggers, and
 network traffic analysis.
Whenever possible, I have relied exclusively on official Niantic communications
 and my own research.
I have noted all data not personally validated.\\
\\
\noindent{}Changes are frequently made to the game, some of them quite fundamental.
This text documents PGO as it is played in July 2025.
I have not generally explored variances with historical gameplay.\\
\\
\noindent{}Mathematical sophistication is not required.
A basic familiarity with algebra ought suffice.
I resort to calculus only once.
I will not be recommending Pokémon, nor teams thereof---ephemeral information at best---but
 instead giving you the tools to make your own creative decisions.\\
\\
\noindent{}Much of the information herein will be old news to experienced
 PGO players, but I hope to provide both a rigorous (if opinionated) reference and an ordered collection of wisdom.
The community established a large body of knowledge long before I
 began playing in March of 2025.
Still, some of my theories regarding team selection and battle strategy might
  be new to you.
My novel contributions are primarily found in \autoref{chap:unbounded},
  \autoref{chap:bounded}, and \autoref{chap:simul}.\\
\\
\noindent{}While writing this book, I extensively consulted the
  GamePress Pokémon GO Wiki\footnote{\url{https://pokemongo.gamepress.gg}},
  the Fandom Pokémon GO Wiki\footnote{\url{https://pokemongo.fandom.com/}},
  the Pokémon GO Database\footnote{\url{https://db.pokemongohub.net/}},
  and the work of the now-disbanded Silph Road research group.
In particular, all bitmaps were sourced from the two wikis.
All graphs, diagrams, and text are my original work.\\
\\
\noindent{}Please send corrections and feedback to 
  \href{mailto:nickblack@linux.com}{nickblack@linux.com}.
\mainmatter
\chapter{Objectives}
\label{sec:goal}
The goal of Pokémon GO is to consume time and divert one's attentions,
  ideally triggering regular releases of dopamine in your ratlike brain.
A secondary objective is stunting on webfora adolescents, but one
  can simply lie, making the game something of an unnecessary complication.
Would you like to be hit by a car? Brother, if so, this is the game for you.\\
\\
Like the Halting Problem, there is no necessary end condition.
Eventually most of the matmuls for which you are responsible will be performed in a data center.
You'll receive UBI, perhaps, I guess, for some reason, or simply be transferred to the Professor.
Either way, you can shove cute iconography into your fat face until you fucking die.\\
\\
The only winning move is not to play, but one could after all say the same
  thing about life itself.\\
\\
Let's begin!
